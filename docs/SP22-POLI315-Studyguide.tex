% Options for packages loaded elsewhere
\PassOptionsToPackage{unicode}{hyperref}
\PassOptionsToPackage{hyphens}{url}
%
\documentclass[
]{book}
\usepackage{amsmath,amssymb}
\usepackage{lmodern}
\usepackage{iftex}
\ifPDFTeX
  \usepackage[T1]{fontenc}
  \usepackage[utf8]{inputenc}
  \usepackage{textcomp} % provide euro and other symbols
\else % if luatex or xetex
  \usepackage{unicode-math}
  \defaultfontfeatures{Scale=MatchLowercase}
  \defaultfontfeatures[\rmfamily]{Ligatures=TeX,Scale=1}
\fi
% Use upquote if available, for straight quotes in verbatim environments
\IfFileExists{upquote.sty}{\usepackage{upquote}}{}
\IfFileExists{microtype.sty}{% use microtype if available
  \usepackage[]{microtype}
  \UseMicrotypeSet[protrusion]{basicmath} % disable protrusion for tt fonts
}{}
\makeatletter
\@ifundefined{KOMAClassName}{% if non-KOMA class
  \IfFileExists{parskip.sty}{%
    \usepackage{parskip}
  }{% else
    \setlength{\parindent}{0pt}
    \setlength{\parskip}{6pt plus 2pt minus 1pt}}
}{% if KOMA class
  \KOMAoptions{parskip=half}}
\makeatother
\usepackage{xcolor}
\IfFileExists{xurl.sty}{\usepackage{xurl}}{} % add URL line breaks if available
\IfFileExists{bookmark.sty}{\usepackage{bookmark}}{\usepackage{hyperref}}
\hypersetup{
  pdftitle={Style Guide for POLI315 International Relations in Spring 2022},
  pdfauthor={Sanghoon Park},
  hidelinks,
  pdfcreator={LaTeX via pandoc}}
\urlstyle{same} % disable monospaced font for URLs
\usepackage{longtable,booktabs,array}
\usepackage{calc} % for calculating minipage widths
% Correct order of tables after \paragraph or \subparagraph
\usepackage{etoolbox}
\makeatletter
\patchcmd\longtable{\par}{\if@noskipsec\mbox{}\fi\par}{}{}
\makeatother
% Allow footnotes in longtable head/foot
\IfFileExists{footnotehyper.sty}{\usepackage{footnotehyper}}{\usepackage{footnote}}
\makesavenoteenv{longtable}
\usepackage{graphicx}
\makeatletter
\def\maxwidth{\ifdim\Gin@nat@width>\linewidth\linewidth\else\Gin@nat@width\fi}
\def\maxheight{\ifdim\Gin@nat@height>\textheight\textheight\else\Gin@nat@height\fi}
\makeatother
% Scale images if necessary, so that they will not overflow the page
% margins by default, and it is still possible to overwrite the defaults
% using explicit options in \includegraphics[width, height, ...]{}
\setkeys{Gin}{width=\maxwidth,height=\maxheight,keepaspectratio}
% Set default figure placement to htbp
\makeatletter
\def\fps@figure{htbp}
\makeatother
\setlength{\emergencystretch}{3em} % prevent overfull lines
\providecommand{\tightlist}{%
  \setlength{\itemsep}{0pt}\setlength{\parskip}{0pt}}
\setcounter{secnumdepth}{5}
\usepackage{booktabs}
\usepackage{hyperref}
\hypersetup{colorlinks,%
citecolor=blue,%
linkcolor=blue,%
urlcolor=blue,%
pdftex}

\usepackage{amsmath}
\usepackage{booktabs}
\usepackage{caption}
\usepackage{longtable}
\ifLuaTeX
  \usepackage{selnolig}  % disable illegal ligatures
\fi
\usepackage[]{natbib}
\bibliographystyle{apalike}

\title{Style Guide for POLI315 International Relations in Spring 2022}
\author{Sanghoon Park}
\date{2022-04-13}

\begin{document}
\maketitle

{
\setcounter{tocdepth}{1}
\tableofcontents
}
\hypertarget{preface}{%
\chapter*{Preface}\label{preface}}
\addcontentsline{toc}{chapter}{Preface}

The exams will evaluate student's knowledge of key concepts discussed in class. The midterm exam will cover material from classes between January 13th and March 3rd, while the final exam will cover material from classes between March 22nd and April 21st.

\hypertarget{final-description}{%
\section*{Final Description}\label{final-description}}
\addcontentsline{toc}{section}{Final Description}

\hypertarget{date-and-time}{%
\subsection*{Date and Time}\label{date-and-time}}
\addcontentsline{toc}{subsection}{Date and Time}

Thursday, April 28th from 4:00 pm to 5:15 pm

\hypertarget{items}{%
\subsection*{Items}\label{items}}
\addcontentsline{toc}{subsection}{Items}

Final will consist of multiple choice questions, true/false questions, and a short answers.

\hypertarget{make-up}{%
\subsection*{Make-up}\label{make-up}}
\addcontentsline{toc}{subsection}{Make-up}

Makeup final exams will be allowed only with pre-approval of the instructor or with an acceptable, documented reason. Acceptable reasons for makeup exams include severe illness, family emergencies or other unavoidable events including dangerous weather conditions and car accidents. Exam format for makeup exams may be different from the original exam and will likely utilize a short answer format. An oral examination may also be utilized if deemed appropriate by the instructor.

\hypertarget{final-format}{%
\subsection*{Final Format}\label{final-format}}
\addcontentsline{toc}{subsection}{Final Format}

\captionsetup[table]{labelformat=empty,skip=1pt}
\begin{longtable}{lrrr}
\caption*{
{\large SPRING 2022 POLI315 Final}
} \\ 
\toprule
Type & Number & Point for individual question (pt) & Sum (pt) \\ 
\midrule
Multiple Choices & 15 & 3 & 45 \\ 
T/F & 15 & 3 & 45 \\ 
Short answers & 2 & 5 & 10 \\ 
Total & 32 &  & 100 \\ 
 \bottomrule
\end{longtable}

As the Final is worth of 25\% in final grades, I will calculate the score of Final as 0.25 \(\times\) Final points. For example, if you obtain 80 pts in the final, it will be 20 pts in your final grades.

\hypertarget{how-to-study}{%
\section*{How to study}\label{how-to-study}}
\addcontentsline{toc}{section}{How to study}

Remember our course objectives:

\begin{itemize}
\item
  As a result of the class, students will be able to:

  \begin{itemize}
  \item
    have some basic knowledge of the definition of international relations and the different approaches to studying international relations.
  \item
    identify and describe dominant topics and concepts related to international relations.
  \item
    obtain a comprehensive understanding of international relations.
  \item
    sharpen research and critical thinking skills.
  \end{itemize}
\item
  Thus, students should:

  \begin{itemize}
  \item
    know basic concepts to understand the topics in international relations.
  \item
    know existing theories to explain the phenomenon in which scholars in international relations are interested.
  \item
    be able to apply the existing theories to different contexts.
  \end{itemize}
\end{itemize}

\hypertarget{international-laws-intergovernmental-organizations}{%
\chapter{International Laws \& Intergovernmental Organizations}\label{international-laws-intergovernmental-organizations}}

\hypertarget{international-cooperation-and-laws}{%
\section{International Cooperation and Laws}\label{international-cooperation-and-laws}}

\hypertarget{international-cooperation}{%
\subsection{International cooperation}\label{international-cooperation}}

Actors mutually adjust their behavior to accommodate the actual or anticipated preferences of others in the pursuit of common goals. Their preferences are not identical and irreconcilable. International cooperation exists when states adopt behavior consistent with the preferences of other states to achieve common objectives like avoiding war, reconciling trade imbalances, or stopping the proliferation of cybercrime.

\hypertarget{what-makes-cooperation-difficult}{%
\subsection{What makes cooperation difficult?}\label{what-makes-cooperation-difficult}}

\hypertarget{relative-gains}{%
\subsubsection{Relative gains}\label{relative-gains}}

Refers to how much more one state gains over another from a given interaction.

\begin{itemize}
\item
  The benefits of cooperation are unlikely to be evenly distributed among participating states.
\item
  States can be hesitant to cooperate when one side reaps larger benefits of cooperation.
\end{itemize}

\hypertarget{prisoners-dilemma}{%
\subsubsection{Prisoner's dilemma}\label{prisoners-dilemma}}

Cooperation is a risky maneuver in the face of the potential for cheating by others.

\begin{itemize}
\item
  Knowing that the incentive to cheat exists, each side is driven away from the choice to cooperate.
\item
  Anarchy forces states to make choices based solely on their self-interests, making cooperation very difficult.
\end{itemize}

\hypertarget{example}{%
\paragraph{Example}\label{example}}

Adam and Bob have robbed a bank and been arrested.

\begin{itemize}
\item
  They are interrogated separately.
\item
  Adam and Bob have the option

  \begin{enumerate}
  \def\labelenumi{\arabic{enumi}.}
  \item
    to confess (move \texttt{C}) or
  \item
    to remain silent (move \texttt{S}).
  \end{enumerate}
\end{itemize}

The police have little evidence, and if both remain silent they will be sentenced to one year on a minor charge. Therefore the police interrogators propose a deal:

\begin{itemize}
\item
  If one confesses while the other remains silent, the one confessing goes free while the other is sentenced to three years.
\item
  However, if both talk, both will still be sentenced to two years.
\end{itemize}

In this game, if each player's payoff is 3 minus the number of years served in jail, we get the following payoff bimatrix.

\begin{table}
\centering
\begin{tabular}[t]{>{\raggedright\arraybackslash}p{10em}|>{\raggedright\arraybackslash}p{10em}|>{\raggedright\arraybackslash}p{10em}}
\hline
  & S & C\\
\hline
.bold[S] & 2, 2 & 0, 3\\
\hline
.bold[C] & 3, 0 & 1, 1\\
\hline
\end{tabular}
\end{table}

It seems obvious that both should remain silent, but that's not likely to happen.

\begin{itemize}
\item
  Each player's move \texttt{C} strictly dominates move \texttt{S}.
\item
  Furthermore, the best response to move S is \texttt{C}, and the best response to move \texttt{C} is also move \texttt{C}, therefore the pair (\texttt{C}, \texttt{C})---both confessing forms the unique .bold{[}Nash equilibrium{]} of this game.
\item
  The choice \texttt{C} ---confessing---with payoffs of only 1 may seem counterintuitive if negotiations can take place in advance, but their terms are non-binding and cannot be enforced.
\item
  It would be useless to agree on move \texttt{S} in advance, since each of the players would feel a strong urge to deviate (cheat). Only if binding agreements are possible, would both agree on the \texttt{S}-\texttt{S} combination, reaching a higher payoff.
\end{itemize}

Thus \textbf{PRISONER'S DILEMMA} gives a paradoxical result.

\begin{itemize}
\item
  Players will play moves that result in lower payoffs for both than are possible.
\item
  This is in part because the rules of the game do not allow binding agreements.
\end{itemize}

\hypertarget{why-do-some-believe-international-cooperation-is-likely}{%
\subsection{Why do some believe international cooperation is likely?}\label{why-do-some-believe-international-cooperation-is-likely}}

Despite systemic anarchy, states often cooperate because cooperation may be in their self-interest.

\begin{itemize}
\item
  States interact continuously and can reciprocate both cooperation and cheating.
\item
  The expectation of reciprocity makes cooperation a rational choice.
\item
  Shadow of the future: states expect that they will have repeated interactions.

  \begin{itemize}
  \tightlist
  \item
    For example, the United States--Mexico--Canada trade agreement creates expectations of consistent future interactions.
  \end{itemize}
\end{itemize}

If the prisoner's dilemma is played only once, it is in each prisoner's self-interest to defect. However, if the prisoner's dilemma interaction is played repeatedly, the likelihood of reciprocity (referred to as a tit-for-tat strategy) makes it rational for each prisoner to cooperate rather than defect.

\hypertarget{solution-as-international-institution}{%
\subsubsection{Solution as international institution?}\label{solution-as-international-institution}}

International institutions foster cooperation by encouraging reciprocity.

\begin{itemize}
\item
  Institutions make cooperative and non-cooperative behavior easier to identify.
\item
  Institutions can provide states with information about the behavior of others.
\item
  Institutions enable states to align their expectations about what cooperative behavior looks like.
\end{itemize}

Neoliberal institutionalist suggests institutions as a way to prevent states from cheating

\begin{itemize}
\item
  Setting standards of behavior: Establishing rules explicitly through charters or over time through norms
\item
  Verifying compliance: Monitoring (elections, production of nuclear material)
\item
  Reducing costs of joint decision making: Requires institutions to `stick'
\item
  Resolving Disputes: Resolution mechanisms (WTO, Regional Trade Agreements)
\end{itemize}

\hypertarget{international-laws}{%
\subsection{International Laws}\label{international-laws}}

Consists of a body of rules and norms regulating interactions among states, between states and IGOs, and in more limited cases, among IGOs, states, and individuals. Sources of international law include followings:

\begin{itemize}
\item
  Customs---ingrained habits, usually created by groups of states or a hegemon.
\item
  Treaties---binding written agreements that lay out rights and obligations: Usually require ratification.

  \begin{itemize}
  \tightlist
  \item
    Treaties are crucial in establishing law across all areas of the international system.
  \end{itemize}
\item
  Enforcement Mechanisms and State Compliance

  \begin{itemize}
  \item
    States are sovereign actors, and compliance with international law is clearly not absolute.
  \item
    The U.S. invasion of Iraq (2003) violated the UN Charter.
  \item
    Yet, most of the time, states do comply with international law.
  \end{itemize}
\end{itemize}

\hypertarget{vertical-enforcement}{%
\subsubsection{Vertical Enforcement}\label{vertical-enforcement}}

A legal process whereby one actor works to constrain the actions of another actor over which it has authority in order to secure its compliance with the law.

\hypertarget{horizontal-enforcement}{%
\subsubsection{Horizontal Enforcement}\label{horizontal-enforcement}}

States work to elicit compliance with international law by other states.

\begin{itemize}
\item
  Power: States comply with international law because more powerful states make them.
\item
  Reciprocity/tit for tat: the desire to reap the gains of cooperation will incentivize states to comply with the laws.
\end{itemize}

\hypertarget{intergovernmental-organizations}{%
\section{Intergovernmental Organizations}\label{intergovernmental-organizations}}

International institutions established by states and whose members are the governments of states.

\begin{itemize}
\tightlist
\item
  Examples: the United Nations, OPEC, the European Union, etc.
\end{itemize}

Why do states organize themselves collectively through intergovernmental organizations?

\begin{itemize}
\item
  Neoliberal institutionalism: Continuous interaction among states provides motivation for them to create international organizations, which, in turn:

  \begin{itemize}
  \item
    Moderate state behavior.
  \item
    Provide a framework for interaction.
  \item
    Establish mechanisms to reduce cheating.
  \item
    Facilitate transparency of state actions.
  \item
    When states repeatedly interact with one another, they can reciprocate cooperative actions or punish non-cooperative ones.

    \begin{itemize}
    \tightlist
    \item
      International institutions help to solidify these repeated interactions by creating lasting relationships among states.
    \end{itemize}
  \item
    Particularly useful for solving two types of problems:

    \begin{itemize}
    \item
      Technical, nonpolitical issues for which states are not the best units for problem resolution (functionalism)
    \item
      Management of collective goods: goods available to all group members, regardless of individual contribution
    \end{itemize}
  \end{itemize}
\end{itemize}

\hypertarget{roles-of-igos-the-international-system}{%
\subsection{Roles of IGOs: The International System}\label{roles-of-igos-the-international-system}}

\begin{itemize}
\item
  Contribute to habits of cooperation (European Union)
\item
  Gather information; surveillance (International Atomic Energy Agency)
\item
  Settle disputes (World Trade Organization)
\item
  Conduct operational activities (World Food Programme)
\item
  Arena for bargaining (World Trade Organization)
\item
  Lead to creation of transnational societal networks (European Union)
\end{itemize}

\hypertarget{roles-of-igos-states}{%
\subsection{Roles of IGOs: States}\label{roles-of-igos-states}}

\begin{itemize}
\item
  Expand the possibilities for foreign policy making
\item
  Used by states as instrument of foreign policy: legitimize foreign policy
\item
  Enhance available information
\item
  Punish or constrain state behavior
\end{itemize}

\hypertarget{the-united-nations}{%
\subsection{The United Nations}\label{the-united-nations}}

Founded as the League of Nations after World War I to focus on the notion of collective security. Guided by three principles:

\begin{enumerate}
\def\labelenumi{\arabic{enumi}.}
\item
  Each state is legally the equivalent of every other state.
\item
  Only international problems fall within the jurisdiction of the United Nations.
\item
  The United Nations is designed primarily to maintain international peace and security.
\end{enumerate}

\hypertarget{security-council}{%
\subsubsection{Security Council}\label{security-council}}

\begin{itemize}
\item
  Permanent members (5): the United States, Great Britain, France, Russia, and the People's Republic of China
\item
  Have the ability to veto substantive resolutions passed by the council and ten additional rotating members elected by region.
\item
  Under Chapter VII of the UN Charter, the Security Council has the power to authorize economic sanctions or the use of force against a state that violates international peace and security.
\end{itemize}

\hypertarget{general-assembly}{%
\subsubsection{General Assembly}\label{general-assembly}}

\begin{itemize}
\item
  Forum for states to air ideas and complaints from constituents
\item
  Arena in which member states can debate
\item
  Evaluates and approves the UN budget
\item
  Resolutions can provide the basis for new international laws.
\end{itemize}

Limited influence because the General Assembly can make only recommendations and members have widely diverse interests.

\hypertarget{key-political-issues-for-the-united-nations}{%
\subsubsection{Key Political Issues for the United Nations}\label{key-political-issues-for-the-united-nations}}

\begin{itemize}
\item
  Development of peacekeeping

  \begin{itemize}
  \tightlist
  \item
    Evolved as a way to limit conflict and prevent escalation into Cold War confrontation
  \end{itemize}
\item
  Post--Cold War Chapter VII enforcement
\item
  Continuous efforts to reform
\end{itemize}

\hypertarget{traditional-peacekeeping}{%
\subsubsection{Traditional Peacekeeping}\label{traditional-peacekeeping}}

\begin{itemize}
\item
  Uses third-party military forces drawn from nonpermanent members of the Security Council
\item
  Prevents conflicts from escalating
\item
  Invited in by disputants
\item
  Focuses on separating warring parties (buffer zone), securing borders, patrolling demarcation, maintaining cease-fires
\end{itemize}

\hypertarget{complex-peacekeeping}{%
\subsection{Complex Peacekeeping}\label{complex-peacekeeping}}

Also known as multidimensional peacekeeping:

\begin{itemize}
\item
  Respond to civil wars, ethnonational conflicts, and domestic unrest
\item
  Disputants may not have requested UN assistance
\item
  Use of military and civilian personnel (including those drawn from the Security Council)

  \begin{itemize}
  \item
    Verifying troop withdrawals
  \item
    Separating warring factions
  \item
    Conducting and supervising elections
  \item
    Implementing human rights guarantees
  \item
    Supplying humanitarian aid
  \item
    Helping civil administration maintain law and order (also known as peacebuilding)
  \end{itemize}
\end{itemize}

\hypertarget{nongovernmental-organizations-ngos}{%
\subsection{Nongovernmental Organizations (NGOs)}\label{nongovernmental-organizations-ngos}}

Private, voluntary organizations whose members are individuals or associations that come together to address a common purpose, often oriented to a public good.

\begin{itemize}
\item
  Not sovereign; lack resources available to states.
\item
  Some entirely private, and some partially relying on government aid.
\item
  Some are open to mass membership; others are closed-member groups.
\item
  The number of NGOs has grown dramatically.
\end{itemize}

In recent decades, NGOs have grown in importance due to the communications revolution (fax, internet, e-mail, social media) recruit to, and launch the publicity campaigns of many NGOs.

\hypertarget{functions-and-roles-of-ngos}{%
\subsubsection{Functions and Roles of NGOs}\label{functions-and-roles-of-ngos}}

\begin{itemize}
\item
  Advocate for specific policies
\item
  Alternative channel for political participation
\item
  Mobilize mass publics
\item
  Distribute aid
\item
  Monitor norms and state practices
\end{itemize}

\hypertarget{the-power-of-ngos}{%
\subsubsection{The Power of NGOs}\label{the-power-of-ngos}}

NGOs rely on soft power, trying to persuade others to change their behavior. Having an independent donor base and links with grassroots groups provides flexibility of actions.

\begin{itemize}
\tightlist
\item
  Can operate in different areas of the world
\end{itemize}

Being politically independent allows for rapid and direct execution of policy initiatives.

\hypertarget{the-limits-of-ngos}{%
\subsubsection{The Limits of NGOs}\label{the-limits-of-ngos}}

NGOs often lack material forms of power; they cannot command obedience through physical means. Most NGOs have very limited funding. Many NGOs rely on governments, which raises questions of legitimacy and neutrality.

\hypertarget{midterm-exam-pool}{%
\chapter*{Midterm Exam Pool}\label{midterm-exam-pool}}
\addcontentsline{toc}{chapter}{Midterm Exam Pool}

\hypertarget{part-i.-multiple-choice}{%
\section*{Part I. Multiple-Choice}\label{part-i.-multiple-choice}}
\addcontentsline{toc}{section}{Part I. Multiple-Choice}

I will pick 13 MC quizzes from here.

\begin{center}\rule{0.5\linewidth}{0.5pt}\end{center}

\begin{enumerate}
\def\labelenumi{\arabic{enumi}.}
\item
  Which of the following is true of the Treaties of Westphalia?

  A. They sought to break up permanent national militaries, giving rise to the Thirty Years' War.

  B. They made sure that no state or states could dominate the system after the Thirty Years' War.

  C. They created formal international institutions to maintain the balance of power after the Thirty Years' War.

  D. They codified the rights of states to determine their own domestic policies after the Thirty Years' War
\item
  Traditionally, international relations scholars trace the origin of the modern state system to which event?

  \begin{enumerate}
  \def\labelenumii{\Alph{enumii}.}
  \item
    The end of World War II
  \item
    The fall of the Roman Empire
  \item
    The Treaties of Westphalia
  \item
    The Great Depression
  \item
    None of the above.
  \end{enumerate}
\item
  What best summarizes Adam Smith's contribution to economic theory in the 1700s?

  \begin{enumerate}
  \def\labelenumii{\Alph{enumii}.}
  \item
    Smith argued in favor of government regulation of the economy.
  \item
    Smith believed that the Treaties of Westphalia would bring renewed economic prosperity.
  \item
    Smith introduced the concept of `Comparative advantage,' meaning an economy's ability to produce a particular good or service at a lower opportunity cost than its trading partners.
  \item
    Smith united two key concepts: division of labor as a motor for generating prosperity, and market systems based on self-interest as a fuel for that motor.
  \item
    None of the above.
  \end{enumerate}
\item
  Which of the following is true of the core group of European states that dominated from the end of the Treaties of Westphalia to the beginning of the nineteenth century?

  A. All the core states experienced an economic revival because they were democracies.

  B. None of the core states experienced an economic revival, as liberal capitalism failed at the time.

  C. Only the core states in Western Europe (England, France, and the United Provinces) experienced an economic revival because they were democracies.

  D. The core states in Western Europe (England, France, and the United Provinces) experienced an economic revival under liberal capitalism.

  E. None of above.
\item
  As part of the nineteenth-century balance-of-power system in Europe,

  A. independent states balanced colonies of relatively equal power.

  B. treaties were designed to create the emergence of a hegemon.

  C. alliances were formed to counteract potentially more powerful factions.

  D. agricultural elites balanced against urban factory owners.
\item
  What is true about industrialization and political power in Europe in the 1800s?

  \begin{enumerate}
  \def\labelenumii{\Alph{enumii}.}
  \item
    Industrialization did not occur in such places as England and Germany.
  \item
    Wealthy landowners gained more power from industrialization.
  \item
    Industrialization favored the middle classes at the expense of aristocrats.
  \item
    Farmers gained political power at the expense of factory workers.
  \item
    None of the above
  \end{enumerate}
\item
  Assuming a state is a(n) \_\_\_\_\_\_\_\_\_\_ means that it has its own defined interests and chooses its own actions, whereas assuming a state is a(n) \_\_\_\_\_\_\_\_\_\_ means that it makes decisions by weighing the costs and benefits of various options.

  \begin{enumerate}
  \def\labelenumii{\Alph{enumii}.}
  \item
    irrational actor; rational actor
  \item
    unitary actor; irrational actor
  \item
    rational actor; unitary actor
  \item
    unitary actor; rational actor
  \end{enumerate}
\item
  Which of the following illustrates the security dilemma?

  A. India and Pakistan will not cooperate since they have relatively little to gain from doing so.

  B. India's increase in its nuclear arms led Pakistan to be less secure, leading Pakistan to increase its arms as well.

  C. India and Pakistan are likely to cooperate only when one is more powerful than the other.

  D. If India attacked Pakistan, other states will come to Pakistan's aid.

  E. None of above.
\item
  Conceptualizing international relations as a ``system'' suggests that

  A. states and other relevant actors are automatons that lack the capacity to change.

  B. there are no regular patterns to international affairs.

  C. actors involved in international relations interact with each other in regularized ways.

  D. you can only study the whole of international politics at once and not focus on individual actions.
\item
  Which of the following is true of a bipolar system?

  \begin{enumerate}
  \def\labelenumii{\Alph{enumii}.}
  \item
    International organizations are likely to be ineffective because they cannot direct the behavior of either of the poles.
  \item
    Alliances are fluid to preserve the relative balance.
  \item
    Constructivists focus on the importance of the bipolar system but not the others.
  \item
    Neo-realists think that bipolar systems are the least stable.
  \item
    None of above.
  \end{enumerate}
\item
  Which of the following is a norm of the balance-of-power (multipolar) system?

  A. Alliances are formed for a specific purpose and a short duration and shift according to advantage.

  B. Alliances should be fixed.

  C. States should align together based on their economic system.

  D. Some states are ruled out as potential allies.
\item
  The idea that informal meetings, such as the one between heads of state Chancellor Angela Merkel of Germany and Prime Minister Narendra Modi of India at Merkel's country retreat in 2017, can help to foster cooperation between those states is an argument consistent with

  \begin{enumerate}
  \def\labelenumii{\Alph{enumii}.}
  \item
    The Beijing Consensus
  \item
    The concept of complex interdependence
  \item
    The concept of the democratic peace
  \item
    The Washington consensus
  \end{enumerate}
\item
  The fact that, after being labeled as a member of the ``axis of evil'' by the U.S. president, Iran changed its policies from working with the United States post--September 11, 2001, to working against it best exemplifies the

  \begin{enumerate}
  \def\labelenumii{\Alph{enumii}.}
  \item
    realist argument that interests do not matter.
  \item
    constructivist argument that discourse can change state behavior.
  \item
    liberal argument that complex interdependence takes a long time to foster cooperation.
  \item
    dependency theorists' argument that the more powerful developing countries will never allow themselves to be dominated by developed countries.
  \end{enumerate}
\item
  According to constructivists,

  A. systems are set by material structures alone.

  B. systems are unchanging over time.

  C. anarchy can lead to different outcomes at different times.

  D. stratification structures systems.
\item
  After having been defeated in World War I, Germany rose in power and almost achieved parity with France and Britain. That this led Germany to act militarily to secure its new position in the international system would be an argument made by a

  \begin{enumerate}
  \def\labelenumii{\Alph{enumii}.}
  \item
    neorealist
  \item
    power balancing theorist
  \item
    power transition theorist
  \item
    hegemonic stability theorist
  \item
    dependence theorist
  \end{enumerate}
\item
  Which of the following statements is true of the study of individuals in international relations?

  A. Realists believe the individual is the most appropriate level of analysis.

  B. Even individuals who are not state leaders can have a significant influence on war and peace.

  C. The extent to which individuals matter is the same across all IR theories.

  D. There is no empirical evidence that individual leaders and their personal characteristics make a difference in foreign policy.
\item
  Which of the following is true of the role of individuals in foreign policy making?

  \begin{enumerate}
  \def\labelenumii{\Alph{enumii}.}
  \item
    Public opinion is not likely to influence foreign policy.
  \item
    Non-elite individuals can and do sometimes play key roles in foreign policy making.
  \item
    The personality characteristics of individual leaders cannot matter in democracies because they must answer to the public.
  \item
    The mass public cannot influence foreign policy through direct actions.
  \item
    None of the above.
  \end{enumerate}
\item
  The ``two-level game'' of international negotiation refers to

  A. the fact that states must often negotiate not only with an opponent state, but third-party states as well.

  B. the fact that bargaining occurs between states as well as between state negotiators and their various domestic constituencies.

  C. when diplomatic negotiations are coupled with economic statecraft.

  D. the use of a mediator in negotiations.
\item
  Liberals view the two-level game of international negotiations as

  \begin{enumerate}
  \def\labelenumii{\Alph{enumii}.}
  \item
    highlighting the importance of domestic politics in international relations.
  \item
    showing that domestic politics is not important in international relations.
  \item
    being constrained by the structure of the international system.
  \item
    showing that the structure of the international system is not important in international relations.
  \end{enumerate}
\item
  Which of the following is true of the realist view of diplomacy?

  A. Realists believe that diplomacy is the most effective tool of statecraft.

  B. Realists believe that diplomacy is likely to be ineffective without being backed by economic statecraft or force.

  C. Realists believe that public diplomacy can be effective, but not Track Two diplomacy.

  D. Realists believe that diplomacy is effective only if coupled with deterrence.
\item
  What is one characteristic necessary for international relations scholars to define something as a war?

  A. There is random violence.

  B. At least three countries are fighting.

  C. A clear victor emerges.

  D. At least 1,000 deaths occur in a calendar year.
\item
  A key characteristic of all forms of terrorism is that

  A. they are carried out only in the Middle East.

  B. they are religious in nature and intent.

  C. they are a very recent form of warfare.

  D. their essence is psychological, not physical.
\item
  Which of the following is considered a form of conventional warfare?

  \begin{enumerate}
  \def\labelenumii{\Alph{enumii}.}
  \item
    Terrorism
  \item
    Cyber attacks
  \item
    Guerilla tactics
  \item
    Bombing Raids
  \item
    Weapons of mass destruction
  \end{enumerate}
\item
  According to constructivists, wars occur because

  \begin{enumerate}
  \def\labelenumii{\Alph{enumii}.}
  \item
    some state's identities and interests lead them to be aggressive.
  \item
    of the distribution of power.
  \item
    state's are not interdependent.
  \item
    of the security dilemma.
  \item
    None of above
  \end{enumerate}
\item
  Which of the following was not an example used in class for why Civil Wars occur?

  \begin{enumerate}
  \def\labelenumii{\Alph{enumii}.}
  \item
    The size of the country
  \item
    Whether or not the country is an oil exporter
  \item
    Population
  \item
    Geographic features
  \item
    Lack technological advancement
  \end{enumerate}
\item
  Which of the following is a key characteristic of the anarchic international system in relation to realists' understanding of war?

  A. Anarchy has led nondemocracies to become more prevalent than democracies, and nondemocracies are more aggressive than democracies.

  B. Under the condition of anarchy there are few rules about how to decide among contending claims and no effective arbiter to do so, and this can lead to war.

  C. Anarchy leads states to have weak armies, so they are attacked by terrorists.

  D. Realists do not believe that war is the result of the anarchy of the international system.
\end{enumerate}

\begin{quote}
Also, you should read Snyder (2009) and Keohane and Martin (1995).
\end{quote}

\hypertarget{part-ii.-truefalse-question}{%
\section*{Part II. True/False Question}\label{part-ii.-truefalse-question}}
\addcontentsline{toc}{section}{Part II. True/False Question}

I will pick 5 T/F quizzes from here.

\begin{center}\rule{0.5\linewidth}{0.5pt}\end{center}

\begin{enumerate}
\def\labelenumi{\arabic{enumi}.}
\item
  As long as a state has the incentive to carry out a threat, that threat is credible even if the state does not have the ability to follow through.

  A. True

  B. False
\item
  The wars that followed the Arab Spring in 2011 are not categorized as civil wars because the government had significantly more capacity to harm the rebels than the rebels had to harm the government.

  A. True

  B. False
\item
  Formal organizations, such as the United Nations and the North Atlantic Treaty Organization, and treaties such as the Law of the Sea Treaty are all examples of international institutions.

  \begin{enumerate}
  \def\labelenumii{\Alph{enumii}.}
  \item
    True
  \item
    False
  \end{enumerate}
\item
  Great-power competition characterized Europe in the nineteenth century and the relationship between the United States and Soviet Union during the Cold War, but in the twenty-first century, great-power competition is no longer relevant.

  \begin{enumerate}
  \def\labelenumii{\Alph{enumii}.}
  \item
    True
  \item
    False
  \end{enumerate}
\item
  An international institution can refer to a formal organization or to a treaty.

  \begin{enumerate}
  \def\labelenumii{\Alph{enumii}.}
  \item
    True
  \item
    False
  \end{enumerate}
\item
  Following the creation of the Nuclear Nonproliferation Treaty, several states that previously had nuclear weapons dismantled their programs.

  \begin{enumerate}
  \def\labelenumii{\Alph{enumii}.}
  \item
    True
  \item
    False
  \end{enumerate}
\item
  Great-power competition characterized Europe in the nineteenth century and the relationship between the United States and Soviet Union during the Cold War, but in the twenty-first century, great-power competition is no longer relevant.

  \begin{enumerate}
  \def\labelenumii{\Alph{enumii}.}
  \item
    True
  \item
    False
  \end{enumerate}
\item
  Constructivists like Alexander Wendt argue that knowing the distribution of material capabilities in the international system is enough to predict whether two states will be friends or foes.

  \begin{enumerate}
  \def\labelenumii{\Alph{enumii}.}
  \item
    True
  \item
    False
  \end{enumerate}
\item
  In bipolar systems, alliances are long-term relationships based on interests, whereas in multipolar balance-of-power systems, alliances are short-term relationships formed for a specific purpose.

  \begin{enumerate}
  \def\labelenumii{\Alph{enumii}.}
  \item
    True
  \item
    False
  \end{enumerate}
\item
  Globalization has promoted both cultural homogenization and differentiation.

  \begin{enumerate}
  \def\labelenumii{\Alph{enumii}.}
  \item
    True
  \item
    False
  \end{enumerate}
\end{enumerate}

\hypertarget{part-iii.-short-answers}{%
\section*{Part III. Short answers}\label{part-iii.-short-answers}}
\addcontentsline{toc}{section}{Part III. Short answers}

\begin{quote}
Short-Answer Question Response: \emph{Define} and \emph{explain} the significance of the following terms. Your answers \emph{should be concise} -- typically requiring \emph{no more than three sentences}.
\end{quote}

\begin{quote}
You should answer five of the eight questions below:
\end{quote}

\begin{center}\rule{0.5\linewidth}{0.5pt}\end{center}

\begin{enumerate}
\def\labelenumi{\arabic{enumi}.}
\item
  Polarity
\item
  Anarchy
\item
  Complex interdependence
\item
  Levels of analysis
\item
  Security dilemma
\item
  Collective security
\item
  Treaties of Westphalia
\end{enumerate}

  \bibliography{book.bib,packages.bib}

\end{document}
