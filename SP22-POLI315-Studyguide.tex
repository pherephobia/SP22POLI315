% Options for packages loaded elsewhere
\PassOptionsToPackage{unicode}{hyperref}
\PassOptionsToPackage{hyphens}{url}
%
\documentclass[
]{book}
\usepackage{amsmath,amssymb}
\usepackage{lmodern}
\usepackage{iftex}
\ifPDFTeX
  \usepackage[T1]{fontenc}
  \usepackage[utf8]{inputenc}
  \usepackage{textcomp} % provide euro and other symbols
\else % if luatex or xetex
  \usepackage{unicode-math}
  \defaultfontfeatures{Scale=MatchLowercase}
  \defaultfontfeatures[\rmfamily]{Ligatures=TeX,Scale=1}
\fi
% Use upquote if available, for straight quotes in verbatim environments
\IfFileExists{upquote.sty}{\usepackage{upquote}}{}
\IfFileExists{microtype.sty}{% use microtype if available
  \usepackage[]{microtype}
  \UseMicrotypeSet[protrusion]{basicmath} % disable protrusion for tt fonts
}{}
\makeatletter
\@ifundefined{KOMAClassName}{% if non-KOMA class
  \IfFileExists{parskip.sty}{%
    \usepackage{parskip}
  }{% else
    \setlength{\parindent}{0pt}
    \setlength{\parskip}{6pt plus 2pt minus 1pt}}
}{% if KOMA class
  \KOMAoptions{parskip=half}}
\makeatother
\usepackage{xcolor}
\IfFileExists{xurl.sty}{\usepackage{xurl}}{} % add URL line breaks if available
\IfFileExists{bookmark.sty}{\usepackage{bookmark}}{\usepackage{hyperref}}
\hypersetup{
  pdftitle={Style Guide for POLI315 International Relations in Spring 2022},
  pdfauthor={Sanghoon Park},
  hidelinks,
  pdfcreator={LaTeX via pandoc}}
\urlstyle{same} % disable monospaced font for URLs
\usepackage{longtable,booktabs,array}
\usepackage{calc} % for calculating minipage widths
% Correct order of tables after \paragraph or \subparagraph
\usepackage{etoolbox}
\makeatletter
\patchcmd\longtable{\par}{\if@noskipsec\mbox{}\fi\par}{}{}
\makeatother
% Allow footnotes in longtable head/foot
\IfFileExists{footnotehyper.sty}{\usepackage{footnotehyper}}{\usepackage{footnote}}
\makesavenoteenv{longtable}
\usepackage{graphicx}
\makeatletter
\def\maxwidth{\ifdim\Gin@nat@width>\linewidth\linewidth\else\Gin@nat@width\fi}
\def\maxheight{\ifdim\Gin@nat@height>\textheight\textheight\else\Gin@nat@height\fi}
\makeatother
% Scale images if necessary, so that they will not overflow the page
% margins by default, and it is still possible to overwrite the defaults
% using explicit options in \includegraphics[width, height, ...]{}
\setkeys{Gin}{width=\maxwidth,height=\maxheight,keepaspectratio}
% Set default figure placement to htbp
\makeatletter
\def\fps@figure{htbp}
\makeatother
\setlength{\emergencystretch}{3em} % prevent overfull lines
\providecommand{\tightlist}{%
  \setlength{\itemsep}{0pt}\setlength{\parskip}{0pt}}
\setcounter{secnumdepth}{5}
\usepackage{booktabs}
\usepackage{hyperref}
\hypersetup{colorlinks,%
citecolor=blue,%
linkcolor=blue,%
urlcolor=blue,%
pdftex}

\usepackage{amsmath}
\usepackage{booktabs}
\usepackage{caption}
\usepackage{longtable}
\ifLuaTeX
  \usepackage{selnolig}  % disable illegal ligatures
\fi
\usepackage[]{natbib}
\bibliographystyle{apalike}

\title{Style Guide for POLI315 International Relations in Spring 2022}
\author{Sanghoon Park}
\date{2022-04-21}

\begin{document}
\maketitle

{
\setcounter{tocdepth}{1}
\tableofcontents
}
\hypertarget{preface}{%
\chapter*{Preface}\label{preface}}
\addcontentsline{toc}{chapter}{Preface}

The exams will evaluate student's knowledge of key concepts discussed in class. The midterm exam will cover material from classes between January 13th and March 3rd, while the final exam will cover material from classes between March 22nd and April 21st.

\hypertarget{final-description}{%
\section*{Final Description}\label{final-description}}
\addcontentsline{toc}{section}{Final Description}

\hypertarget{date-and-time}{%
\subsection*{Date and Time}\label{date-and-time}}
\addcontentsline{toc}{subsection}{Date and Time}

Thursday, April 28th from 4:00 pm to 5:15 pm

\hypertarget{items}{%
\subsection*{Items}\label{items}}
\addcontentsline{toc}{subsection}{Items}

Final will consist of multiple choice questions, true/false questions, and a short answers.

\hypertarget{make-up}{%
\subsection*{Make-up}\label{make-up}}
\addcontentsline{toc}{subsection}{Make-up}

Makeup final exams will be allowed only with pre-approval of the instructor or with an acceptable, documented reason. Acceptable reasons for makeup exams include severe illness, family emergencies or other unavoidable events including dangerous weather conditions and car accidents. Exam format for makeup exams may be different from the original exam and will likely utilize a short answer format. An oral examination may also be utilized if deemed appropriate by the instructor.

\hypertarget{final-format}{%
\subsection*{Final Format}\label{final-format}}
\addcontentsline{toc}{subsection}{Final Format}

\captionsetup[table]{labelformat=empty,skip=1pt}
\begin{longtable}{lrrr}
\caption*{
{\large SPRING 2022 POLI315 Final}
} \\ 
\toprule
Type & Number & Point for individual question (pt) & Sum (pt) \\ 
\midrule
Multiple Choices & 15 & 3 & 45 \\ 
T/F & 15 & 3 & 45 \\ 
Short Essay & 1 & 10 & 10 \\ 
Total & 31 &  & 100 \\ 
\bottomrule
\end{longtable}

As the Final is worth of 25\% in final grades, I will calculate the score of Final as 0.25 \(\times\) Final points. For example, if you obtain 80 pts in the final, it will be 20 pts in your final grades.

\hypertarget{how-to-study}{%
\section*{How to study}\label{how-to-study}}
\addcontentsline{toc}{section}{How to study}

Remember our course objectives:

\begin{itemize}
\item
  As a result of the class, students will be able to:

  \begin{itemize}
  \item
    have some basic knowledge of the definition of international relations and the different approaches to studying international relations.
  \item
    identify and describe dominant topics and concepts related to international relations.
  \item
    obtain a comprehensive understanding of international relations.
  \item
    sharpen research and critical thinking skills.
  \end{itemize}
\item
  Thus, students should:

  \begin{itemize}
  \item
    know basic concepts to understand the topics in international relations.
  \item
    know existing theories to explain the phenomenon in which scholars in international relations are interested.
  \item
    be able to apply the existing theories to different contexts.
  \end{itemize}
\end{itemize}

\hypertarget{international-laws-intergovernmental-organizations}{%
\chapter{International Laws \& Intergovernmental Organizations}\label{international-laws-intergovernmental-organizations}}

\hypertarget{international-cooperation-and-laws}{%
\section{International Cooperation and Laws}\label{international-cooperation-and-laws}}

\hypertarget{international-cooperation}{%
\subsection{International cooperation}\label{international-cooperation}}

Actors mutually adjust their behavior to accommodate the actual or anticipated preferences of others in the pursuit of common goals. Their preferences are not identical and irreconcilable. International cooperation exists when states adopt behavior consistent with the preferences of other states to achieve common objectives like avoiding war, reconciling trade imbalances, or stopping the proliferation of cybercrime.

\hypertarget{what-makes-cooperation-difficult}{%
\subsection{What makes cooperation difficult?}\label{what-makes-cooperation-difficult}}

\hypertarget{relative-gains}{%
\subsubsection{Relative gains}\label{relative-gains}}

Refers to how much more one state gains over another from a given interaction.

\begin{itemize}
\item
  The benefits of cooperation are unlikely to be evenly distributed among participating states.
\item
  States can be hesitant to cooperate when one side reaps larger benefits of cooperation.
\end{itemize}

\hypertarget{prisoners-dilemma}{%
\subsubsection{Prisoner's dilemma}\label{prisoners-dilemma}}

Cooperation is a risky maneuver in the face of the potential for cheating by others.

\begin{itemize}
\item
  Knowing that the incentive to cheat exists, each side is driven away from the choice to cooperate.
\item
  Anarchy forces states to make choices based solely on their self-interests, making cooperation very difficult.
\end{itemize}

\hypertarget{example}{%
\paragraph{Example}\label{example}}

Adam and Bob have robbed a bank and been arrested.

\begin{itemize}
\item
  They are interrogated separately.
\item
  Adam and Bob have the option

  \begin{enumerate}
  \def\labelenumi{\arabic{enumi}.}
  \item
    to confess (move \texttt{C}) or
  \item
    to remain silent (move \texttt{S}).
  \end{enumerate}
\end{itemize}

The police have little evidence, and if both remain silent they will be sentenced to one year on a minor charge. Therefore the police interrogators propose a deal:

\begin{itemize}
\item
  If one confesses while the other remains silent, the one confessing goes free while the other is sentenced to three years.
\item
  However, if both talk, both will still be sentenced to two years.
\end{itemize}

In this game, if each player's payoff is 3 minus the number of years served in jail, we get the following payoff bimatrix.

\begin{table}
\centering
\begin{tabular}[t]{>{\raggedright\arraybackslash}p{10em}|>{\raggedright\arraybackslash}p{10em}|>{\raggedright\arraybackslash}p{10em}}
\hline
  & S & C\\
\hline
.bold[S] & 2, 2 & 0, 3\\
\hline
.bold[C] & 3, 0 & 1, 1\\
\hline
\end{tabular}
\end{table}

It seems obvious that both should remain silent, but that's not likely to happen.

\begin{itemize}
\item
  Each player's move \texttt{C} strictly dominates move \texttt{S}.
\item
  Furthermore, the best response to move S is \texttt{C}, and the best response to move \texttt{C} is also move \texttt{C}, therefore the pair (\texttt{C}, \texttt{C})---both confessing forms the unique .bold{[}Nash equilibrium{]} of this game.
\item
  The choice \texttt{C} ---confessing---with payoffs of only 1 may seem counterintuitive if negotiations can take place in advance, but their terms are non-binding and cannot be enforced.
\item
  It would be useless to agree on move \texttt{S} in advance, since each of the players would feel a strong urge to deviate (cheat). Only if binding agreements are possible, would both agree on the \texttt{S}-\texttt{S} combination, reaching a higher payoff.
\end{itemize}

Thus \textbf{PRISONER'S DILEMMA} gives a paradoxical result.

\begin{itemize}
\item
  Players will play moves that result in lower payoffs for both than are possible.
\item
  This is in part because the rules of the game do not allow binding agreements.
\end{itemize}

\hypertarget{why-do-some-believe-international-cooperation-is-likely}{%
\subsection{Why do some believe international cooperation is likely?}\label{why-do-some-believe-international-cooperation-is-likely}}

Despite systemic anarchy, states often cooperate because cooperation may be in their self-interest.

\begin{itemize}
\item
  States interact continuously and can reciprocate both cooperation and cheating.
\item
  The expectation of reciprocity makes cooperation a rational choice.
\item
  Shadow of the future: states expect that they will have repeated interactions.

  \begin{itemize}
  \tightlist
  \item
    For example, the United States--Mexico--Canada trade agreement creates expectations of consistent future interactions.
  \end{itemize}
\end{itemize}

If the prisoner's dilemma is played only once, it is in each prisoner's self-interest to defect. However, if the prisoner's dilemma interaction is played repeatedly, the likelihood of reciprocity (referred to as a tit-for-tat strategy) makes it rational for each prisoner to cooperate rather than defect.

\hypertarget{solution-as-international-institution}{%
\subsubsection{Solution as international institution?}\label{solution-as-international-institution}}

International institutions foster cooperation by encouraging reciprocity.

\begin{itemize}
\item
  Institutions make cooperative and non-cooperative behavior easier to identify.
\item
  Institutions can provide states with information about the behavior of others.
\item
  Institutions enable states to align their expectations about what cooperative behavior looks like.
\end{itemize}

Neoliberal institutionalist suggests institutions as a way to prevent states from cheating

\begin{itemize}
\item
  Setting standards of behavior: Establishing rules explicitly through charters or over time through norms
\item
  Verifying compliance: Monitoring (elections, production of nuclear material)
\item
  Reducing costs of joint decision making: Requires institutions to `stick'
\item
  Resolving Disputes: Resolution mechanisms (WTO, Regional Trade Agreements)
\end{itemize}

\hypertarget{international-laws}{%
\subsection{International Laws}\label{international-laws}}

Consists of a body of rules and norms regulating interactions among states, between states and IGOs, and in more limited cases, among IGOs, states, and individuals. Sources of international law include followings:

\begin{itemize}
\item
  Customs---ingrained habits, usually created by groups of states or a hegemon.
\item
  Treaties---binding written agreements that lay out rights and obligations: Usually require ratification.

  \begin{itemize}
  \tightlist
  \item
    Treaties are crucial in establishing law across all areas of the international system.
  \end{itemize}
\item
  Enforcement Mechanisms and State Compliance

  \begin{itemize}
  \item
    States are sovereign actors, and compliance with international law is clearly not absolute.
  \item
    The U.S. invasion of Iraq (2003) violated the UN Charter.
  \item
    Yet, most of the time, states do comply with international law.
  \end{itemize}
\end{itemize}

\hypertarget{vertical-enforcement}{%
\subsubsection{Vertical Enforcement}\label{vertical-enforcement}}

A legal process whereby one actor works to constrain the actions of another actor over which it has authority in order to secure its compliance with the law.

\hypertarget{horizontal-enforcement}{%
\subsubsection{Horizontal Enforcement}\label{horizontal-enforcement}}

States work to elicit compliance with international law by other states.

\begin{itemize}
\item
  Power: States comply with international law because more powerful states make them.
\item
  Reciprocity/tit for tat: the desire to reap the gains of cooperation will incentivize states to comply with the laws.
\end{itemize}

\hypertarget{intergovernmental-organizations}{%
\section{Intergovernmental Organizations}\label{intergovernmental-organizations}}

International institutions established by states and whose members are the governments of states.

\begin{itemize}
\tightlist
\item
  Examples: the United Nations, OPEC, the European Union, etc.
\end{itemize}

Why do states organize themselves collectively through intergovernmental organizations?

\begin{itemize}
\item
  Neoliberal institutionalism: Continuous interaction among states provides motivation for them to create international organizations, which, in turn:

  \begin{itemize}
  \item
    Moderate state behavior.
  \item
    Provide a framework for interaction.
  \item
    Establish mechanisms to reduce cheating.
  \item
    Facilitate transparency of state actions.
  \item
    When states repeatedly interact with one another, they can reciprocate cooperative actions or punish non-cooperative ones.

    \begin{itemize}
    \tightlist
    \item
      International institutions help to solidify these repeated interactions by creating lasting relationships among states.
    \end{itemize}
  \item
    Particularly useful for solving two types of problems:

    \begin{itemize}
    \item
      Technical, nonpolitical issues for which states are not the best units for problem resolution (functionalism)
    \item
      Management of collective goods: goods available to all group members, regardless of individual contribution
    \end{itemize}
  \end{itemize}
\end{itemize}

\hypertarget{roles-of-igos-the-international-system}{%
\subsection{Roles of IGOs: The International System}\label{roles-of-igos-the-international-system}}

\begin{itemize}
\item
  Contribute to habits of cooperation (European Union)
\item
  Gather information; surveillance (International Atomic Energy Agency)
\item
  Settle disputes (World Trade Organization)
\item
  Conduct operational activities (World Food Programme)
\item
  Arena for bargaining (World Trade Organization)
\item
  Lead to creation of transnational societal networks (European Union)
\end{itemize}

\hypertarget{roles-of-igos-states}{%
\subsection{Roles of IGOs: States}\label{roles-of-igos-states}}

\begin{itemize}
\item
  Expand the possibilities for foreign policy making
\item
  Used by states as instrument of foreign policy: legitimize foreign policy
\item
  Enhance available information
\item
  Punish or constrain state behavior
\end{itemize}

\hypertarget{the-united-nations}{%
\subsection{The United Nations}\label{the-united-nations}}

Founded as the League of Nations after World War I to focus on the notion of collective security. Guided by three principles:

\begin{enumerate}
\def\labelenumi{\arabic{enumi}.}
\item
  Each state is legally the equivalent of every other state.
\item
  Only international problems fall within the jurisdiction of the United Nations.
\item
  The United Nations is designed primarily to maintain international peace and security.
\end{enumerate}

\hypertarget{security-council}{%
\subsubsection{Security Council}\label{security-council}}

\begin{itemize}
\item
  Permanent members (5): the United States, Great Britain, France, Russia, and the People's Republic of China
\item
  Have the ability to veto substantive resolutions passed by the council and ten additional rotating members elected by region.
\item
  Under Chapter VII of the UN Charter, the Security Council has the power to authorize economic sanctions or the use of force against a state that violates international peace and security.
\end{itemize}

\hypertarget{general-assembly}{%
\subsubsection{General Assembly}\label{general-assembly}}

\begin{itemize}
\item
  Forum for states to air ideas and complaints from constituents
\item
  Arena in which member states can debate
\item
  Evaluates and approves the UN budget
\item
  Resolutions can provide the basis for new international laws.
\end{itemize}

Limited influence because the General Assembly can make only recommendations and members have widely diverse interests.

\hypertarget{key-political-issues-for-the-united-nations}{%
\subsubsection{Key Political Issues for the United Nations}\label{key-political-issues-for-the-united-nations}}

\begin{itemize}
\item
  Development of peacekeeping

  \begin{itemize}
  \tightlist
  \item
    Evolved as a way to limit conflict and prevent escalation into Cold War confrontation
  \end{itemize}
\item
  Post--Cold War Chapter VII enforcement
\item
  Continuous efforts to reform
\end{itemize}

\hypertarget{traditional-peacekeeping}{%
\subsubsection{Traditional Peacekeeping}\label{traditional-peacekeeping}}

\begin{itemize}
\item
  Uses third-party military forces drawn from nonpermanent members of the Security Council
\item
  Prevents conflicts from escalating
\item
  Invited in by disputants
\item
  Focuses on separating warring parties (buffer zone), securing borders, patrolling demarcation, maintaining cease-fires
\end{itemize}

\hypertarget{complex-peacekeeping}{%
\subsection{Complex Peacekeeping}\label{complex-peacekeeping}}

Also known as multidimensional peacekeeping:

\begin{itemize}
\item
  Respond to civil wars, ethnonational conflicts, and domestic unrest
\item
  Disputants may not have requested UN assistance
\item
  Use of military and civilian personnel (including those drawn from the Security Council)

  \begin{itemize}
  \item
    Verifying troop withdrawals
  \item
    Separating warring factions
  \item
    Conducting and supervising elections
  \item
    Implementing human rights guarantees
  \item
    Supplying humanitarian aid
  \item
    Helping civil administration maintain law and order (also known as peacebuilding)
  \end{itemize}
\end{itemize}

\hypertarget{nongovernmental-organizations-ngos}{%
\subsection{Nongovernmental Organizations (NGOs)}\label{nongovernmental-organizations-ngos}}

Private, voluntary organizations whose members are individuals or associations that come together to address a common purpose, often oriented to a public good.

\begin{itemize}
\item
  Not sovereign; lack resources available to states.
\item
  Some entirely private, and some partially relying on government aid.
\item
  Some are open to mass membership; others are closed-member groups.
\item
  The number of NGOs has grown dramatically.
\end{itemize}

In recent decades, NGOs have grown in importance due to the communications revolution (fax, internet, e-mail, social media) recruit to, and launch the publicity campaigns of many NGOs.

\hypertarget{functions-and-roles-of-ngos}{%
\subsubsection{Functions and Roles of NGOs}\label{functions-and-roles-of-ngos}}

\begin{itemize}
\item
  Advocate for specific policies
\item
  Alternative channel for political participation
\item
  Mobilize mass publics
\item
  Distribute aid
\item
  Monitor norms and state practices
\end{itemize}

\hypertarget{the-power-of-ngos}{%
\subsubsection{The Power of NGOs}\label{the-power-of-ngos}}

NGOs rely on soft power, trying to persuade others to change their behavior. Having an independent donor base and links with grassroots groups provides flexibility of actions.

\begin{itemize}
\tightlist
\item
  Can operate in different areas of the world
\end{itemize}

Being politically independent allows for rapid and direct execution of policy initiatives.

\hypertarget{the-limits-of-ngos}{%
\subsubsection{The Limits of NGOs}\label{the-limits-of-ngos}}

NGOs often lack material forms of power; they cannot command obedience through physical means. Most NGOs have very limited funding. Many NGOs rely on governments, which raises questions of legitimacy and neutrality.

\hypertarget{human-rights}{%
\chapter{Human Rights}\label{human-rights}}

\hypertarget{what-are-the-human-rights}{%
\section{What Are the Human Rights?}\label{what-are-the-human-rights}}

Each one of us no matter who we are or where we are born is entitled to the same basic rights and freedoms.

\begin{itemize}
\item
  Human rights are not privileges.
\item
  Human rights cannot be granted or revoked.
\item
  Human rights inalienable and universal.
\end{itemize}

\hypertarget{basic-concepts}{%
\subsection{Basic Concepts}\label{basic-concepts}}

\hypertarget{political-and-civil-human-rights}{%
\subsubsection{Political and civil human rights}\label{political-and-civil-human-rights}}

\begin{itemize}
\item
  Rights that states cannot take away (free speech, assembly)
\item
  Freedom of speech and assembly, security, and protection of the press
\item
  No individual should be deprived of these rights.
\item
  Enshrined in historical documents, such as

  \begin{itemize}
  \item
    The Magna Carta, 1215
  \item
    The French Declaration of the Rights of Man, 1789
  \item
    U.S. Bill of Rights, 1791
  \end{itemize}
\end{itemize}

\hypertarget{economic-and-social-human-rights}{%
\subsubsection{Economic and social human rights}\label{economic-and-social-human-rights}}

\begin{itemize}
\item
  Rights that states should provide (health care, jobs)
\item
  Decent education, work, health care, and standard of living
\item
  Rights with a focus on ``the material''
\item
  Heavily influenced by Karl Marx's writings and critical social theorists

  \begin{itemize}
  \tightlist
  \item
    Socialist theorists believe that without these guarantees of socioeconomic rights, political and civil rights are meaningless.
  \end{itemize}
\end{itemize}

\hypertarget{collective-rights}{%
\subsubsection{Collective rights}\label{collective-rights}}

\begin{itemize}
\item
  The rights of the marginalized
\item
  The idea of collective rights is not nearly as universal as these rights are contested within states.
\item
  Broad spectrum of rights that states should provide to minorities and the historically marginalized as well as to the collective:

  \begin{itemize}
  \item
    Consist of rights for refugees, ethnic minorities, women, indigenous peoples, and LGBTQ+ individuals
  \item
    Include the right to development and the right to a clean environment
  \end{itemize}
\item
  Highly contested in states and the international arena
\end{itemize}

\hypertarget{human-rights-as-emerging-international-responsibility}{%
\section{Human Rights as Emerging International Responsibility}\label{human-rights-as-emerging-international-responsibility}}

Geneva Conventions form the core of international humanitarian law.

\begin{itemize}
\item
  I: protection of the wounded in the armed forces (1864)
\item
  II: protection of the wounded at sea (1906)
\item
  III: protection of prisoners of war (1929)
\item
  \textbf{IV: protection of noncombatants during wartime (1949)}
\end{itemize}

Geneva Conventions form the core of international humanitarian law. Slow evolution (almost 80 years!) of human rights until World War II. Contrary to popular belief, the Geneva Conventions do not prohibit the use of weapons of mass destruction. Because the Geneva Conventions are concerned with the protection of human rights. The Hague Convention and the Geneva Protocol govern weapons of mass destruction (related to biological and chemical weapons).

\begin{itemize}
\item
  UN Universal Declaration of Human Rights, 1948

  \begin{itemize}
  \tightlist
  \item
    A statement of human rights aspirations; Though not legally binding, the statement identified 30 human rights principles covering both political and economic rights.
  \end{itemize}
\item
  Two legally binding documents

  \begin{itemize}
  \item
    The International Covenant on Economic, Social, and Cultural Rights
  \item
    The International Covenant on Civil and Political Rights
  \end{itemize}
\item
  These three documents are known as the International Bill of Rights.
\end{itemize}

\hypertarget{two-faced-states-states-as-protectors-of-human-rights}{%
\subsubsection{Two-faced States: States as Protectors of Human Rights}\label{two-faced-states-states-as-protectors-of-human-rights}}

Westphalian tradition suggests that states are primarily responsible for protecting human rights standards within their own jurisdiction. Many liberal democratic states support political and civil rights in their foreign policy. Using diplomacy by tying certain benefits to improvements in human rights.

\begin{itemize}
\item
  Offering trade concessions or increased aid
\item
  Punishing through sanctions
\item
  Example studied: Kim and Kroeger (2017) Rewarding the introduction of multiparty elections. \emph{EJPE}
\end{itemize}

\hypertarget{two-faced-states-states-as-abusers-of-human-rights}{%
\subsubsection{Two-faced States: States as Abusers of Human Rights}\label{two-faced-states-states-as-abusers-of-human-rights}}

Authoritarian or autocratic states are more likely to abuse political and civil rights. Less developed states may be unable or unwilling to meet basic obligations of social and economic rights due to scarce resources or lack of political will. State security often prevails over individual rights. Political-civil rights may be repressed in times of economic strife to divert attention from the economy. Culture and history affect a state's human rights record.

\hypertarget{the-role-of-the-international-communityigos-and-ngos}{%
\subsection{The Role of the International Community---IGOs and NGOs}\label{the-role-of-the-international-communityigos-and-ngos}}

\begin{itemize}
\item
  Set human rights standards (United Nations)
\item
  Monitor standards
\item
  Establish complaint procedures
\item
  Compile reports on state behavior
\item
  Investigate alleged violations
\end{itemize}

IGOs may, at times, respond to egregious humanitarian emergencies (United Nations, states).

\begin{itemize}
\item
  In a few cases, states may use IGOs to respond to egregious humanitarian emergencies.
\item
  So-called humanitarian intervention was used in the crisis in Somalia in 1992.
\end{itemize}

\hypertarget{naming-and-shaming}{%
\subsubsection{Naming and shaming}\label{naming-and-shaming}}

Publicly identifying and stigmatizing the non-compliant actors

\begin{itemize}
\item
  NGO's weapon against repressive regimes
\item
  State - primarily targets or supporters of naming and shaming (passive roles)
\end{itemize}

\hypertarget{enforcement-problems}{%
\subsubsection{Enforcement Problems}\label{enforcement-problems}}

A state's signature on treaties is no guarantee of its willingness or ability to enforce treaty provisions. Monitoring via self-reporting presumes a willingness to comply and be transparent. NGOs play a key role in monitoring. Economic embargoes may not achieve changes in human rights policy and may hurt those whom embargoes are intended to help. Military action may cause unintended casualties.

Does monitoring by IGOs or NGOs through investigations, reports, resolutions, and naming and shaming ultimately make a difference for rights protection? The evidence is mixed. One study of over 400 human rights organizations on shaming governments between 1992 and 2004 found that states targeted by NGOs do improve their human rights practices. But shaming is not enough. Shaming is effective when both domestic NGOs on the ground and advocacy by other third parties and individuals are present.

\hypertarget{specific-human-rights-issues-genocide-and-mass-atrocities}{%
\subsubsection{Specific Human Rights Issues: Genocide and Mass Atrocities}\label{specific-human-rights-issues-genocide-and-mass-atrocities}}

\textbf{Crimes against humanity}: Article 7 of the Rome Statute of the International Criminal Court

\begin{itemize}
\item
  In 2021, the Trump administration, echoed by the Biden administration, labeled the Chinese government as committing genocide against the Uighurs and other minority groups.
\item
  This designation is controversial given conflicting interpretation of what constitutes genocide and whether that designation then puts pressure on states and the international community to take action to stop the systematic abuse.
\end{itemize}

\hypertarget{punishing-the-guilty}{%
\subsubsection{Punishing the Guilty}\label{punishing-the-guilty}}

International Criminal Court (ICC) covers four types of crimes: genocide, crimes against humanity, war crimes, and crimes of aggression.

\begin{itemize}
\item
  No individuals are immune from jurisdiction, including heads of states and military leaders.
\item
  Many African heads of state feel unfairly targeted by the ICC.
\item
  Some states, including the United States, feel that ICC impinges on state sovereignty and refuse to sign the treaty.
\item
  While many supporters see the court as essential for establishing international law and enforcing individual accountability, the short-term impact has not been positive.
\item
  Critics see the failure of the ICC to investigate China's actions against the Uighurs as a vast moral failure.
\end{itemize}

\hypertarget{the-globalization-of-rights-womens-rights-as-human-rights}{%
\subsubsection{The Globalization of Rights: Women's Rights as Human Rights}\label{the-globalization-of-rights-womens-rights-as-human-rights}}

The UN reports that violence against women and girls ``persists at alarmingly high levels.''

Post--World War II emphasis on political and civil rights

1960s--1970s: increasing concern for economic rights

\begin{itemize}
\item
  Women in development movement (WID)
\item
  UN-sponsored conferences on women
\end{itemize}

While NGOs and IGOs attempt to combat the problem, the mainstay of enforcement continues to be at the state level. The best statistical predictor of state peacefulness is not democracy or wealth but the level of physical security for women. The higher the level of violence against women, the more likely the state is involved in interstate and intrastate conflict.

\hypertarget{the-debate-over-humanitarian-intervention-and-r2p}{%
\section{The Debate Over Humanitarian Intervention and R2P}\label{the-debate-over-humanitarian-intervention-and-r2p}}

Military action to stop massive violations of human rights may be just and necessary (humanitarian intervention).

\begin{itemize}
\item
  Contradicts and erodes the Westphalian view of state sovereignty
\item
  Why do we see selective bias?

  \begin{itemize}
  \item
    In the nineteenth century, Europeans used military force to protect Christians in Turkey and the Middle East, though they chose not to protect other religious groups.
  \item
    And European nations did not intervene militarily to stop slavery, though they prohibited their own citizens from participating in the slave trade.
  \end{itemize}
\end{itemize}

\textbf{Responsibility to protect (R2P)}: in cases of massive violations of human rights, when domestic avenues for redress have been exhausted, states have a responsibility to intervene.

\begin{itemize}
\item
  How massive do the violations have to be?
\item
  Are UN Security Council authorizations necessary?
\item
  Could states have ulterior motives?
\end{itemize}

\hypertarget{article-bdm-et-al.-2005-thinking-inside-the-box}{%
\subsection{Article: BDM et al.~(2005) Thinking Inside the Box}\label{article-bdm-et-al.-2005-thinking-inside-the-box}}

BDM et al.~(2005) argues that a gradual, continuous democratization process does not lead to a gradual improvement of human rights. Some aspects of democratization process generate higher level of human rights protection. Thus, BDM et al.~(2005) expects that increasing degrees of democracy do not lead smoothly to improved human rights; rather, it depends on the mix of scores on the various dimensions that compose a particular scale. Democracy is reliable as a means of reducing human rights abuses only when institutional reforms pass thresholds that ensure accountability, thereby translating institutional changes into behavioral changes. BDM et al.~(2005) finds that democratization does not yield significant improvements for human rights until party competition is normalized. Human rights improvements reflect the discontinuous function of democratization. They conclude that multiparty competition is more significant than other dimensions to reduce human rights abuses. Increases at a state level of democracy do not generate greater respect for human rights. Fully democratization significantly improves human rights record.

\begin{quote}
At least read abstract.
\end{quote}

\hypertarget{article-park-et-al.-2019-the-coevaluation-of-human-rights-advocacy}{%
\subsection{Article: Park et al.~(2019) The (co)evaluation of human rights advocacy}\label{article-park-et-al.-2019-the-coevaluation-of-human-rights-advocacy}}

Park et al.~(2019) answers the question of how human rights issues change over time. They argues that advocates adopt issues similar to their existing agenda. Resource limitation led organizations to adopt issues that can earn much attention. Testing with quantitative text analysis, previously identified distinct latent topics may combine together as they being discussed together by human rights organizations. Park et al.~(2019) concludes that human rights issues evolve over time and shows that human rights issues are interdependent with each other and co-evolve over time.

\begin{quote}
At least read abstract.
\end{quote}

\hypertarget{the-politics-of-trade}{%
\chapter{The Politics of Trade}\label{the-politics-of-trade}}

\hypertarget{theoretical-approaches-to-the-international-political-economy}{%
\section{Theoretical Approaches to the International Political Economy}\label{theoretical-approaches-to-the-international-political-economy}}

Economic liberalism: Humans acting rationally, in a self-interested way, leads to benefits for all.

Mercantilism: Economic wealth as an instrument of state power

\begin{itemize}
\tightlist
\item
  Protectionism: Measures to protect one's own economy from foreign competition in the name of national interest
\end{itemize}

Economic nationalism: Economic policies are subservient to the national interest

Economic radicalism: A reaction to the excesses of the colonial period and the Industrial Revolution

\hypertarget{the-role-of-states}{%
\subsection{The Role of States}\label{the-role-of-states}}

States can use a variety of tools to influence domestic and international economic policy.

Macroeconomic polices

\begin{itemize}
\item
  Fiscal policies: affect state budgets by setting spending levels and tax rates
\item
  Monetary policies: control the money supply
\end{itemize}

Microeconomics policies: policies on regulation, subsidies, competition, and antitrust actions

\textbf{Exchange rates}: the price of currency in relation to another (floating/fixed rates)

\begin{itemize}
\item
  The International Classical Gold Standard (gold coin standard), 1873-1914

  \begin{itemize}
  \item
    The gold standard became the basis for the international monetary system after 1873.
  \item
    Adopting and maintaining a singular monetary arrangement encouraged international trade and investment by stabilizing international price relationships and facilitating foreign borrowing.
  \end{itemize}
\item
  Moved to gold bullion standard (1931)

  \begin{itemize}
  \item
    When a large amount of money was required as a result of World War I, countries temporarily stopped converting gold and began to increase the issuance of money.
  \item
    This was a major setback for the gold standard.
  \item
    Furthermore, the gold standard system based on the nine pound collapsed decisively with the onset of the Great Depression.
  \item
    After the WWII, the Bretton Woods system was established after it was decided to adopt the gold standard with the US dollar as the reserve currency.
  \item
    The crucial difference between the classical gold standard system and the modern gold standard system is that the United States exclusively performs gold conversion, rather than each country's central banks performing gold conversion independently.
  \end{itemize}
\item
  Moved to floating exchange rate: Nixon shock (1971)

  \begin{itemize}
  \item
    US issues too much dollars for the Cold War and Vietnamese war, regardless of the amount of Gold that US has.
  \item
    Other countries required US to convert their dollars into golds.
  \item
    1971, Nixon supported suspending the dollar's convertibility into gold.
  \item
    The Nixon Shock effectively led to the end of the Bretton Woods Agreement and the convertibility of U.S. dollars into gold.
  \end{itemize}
\end{itemize}

Trade policies

\begin{itemize}
\item
  \textbf{Tariffs}: taxes on goods and services crossing borders
\item
  \textbf{Nontariff barriers}: restrictions on international trade designed to protect health, safety, or national security
\item
  \textbf{Current accounts}: measure the net border flows between countries of goods, services, government transfers, and income on capital investments
\item
  \textbf{Capital accounts}: describe the flows of capital between countries, including foreign direct investment and portfolio investment in and out
\end{itemize}

\textbf{Balance of payments}: a country's current and capital account balances

\begin{itemize}
\item
  Surplus: the value of exports is greater than the value of imports.
\item
  Deficit: the value of imports is greater than the value of exports.
\end{itemize}

\hypertarget{the-role-of-multinational-corporations}{%
\subsection{The Role of Multinational Corporations}\label{the-role-of-multinational-corporations}}

MNCs span state borders through trade and investment and/or actual presence. There are about 60,000 MNCs; they account for 50 percent of worldwide trade. Key engines of global economic growth. Transmission belt for capital, ideas, economic interdependence, and growth.

\hypertarget{bretton-woods-institutions}{%
\section{Bretton Woods Institutions}\label{bretton-woods-institutions}}

A set of intergovernmental organizations to support economic liberalism.

\begin{itemize}
\item
  The World Bank
\item
  The International Monetary Fund (IMF)
\item
  The General Agreement on Tariffs and Trade (GATT) \(\rightarrow\) The World Trade Organization (WTO)
\end{itemize}

\hypertarget{world-bank}{%
\subsection{World Bank}\label{world-bank}}

During the 1950s, the World Bank shifted its emphasis from reconstruction to development.

\begin{itemize}
\item
  Lends funds, with interest, for economic development projects
\item
  Lending is designed not to replace private capital but to facilitate the use of private capital.
\item
  Much of its funding has been used for infrastructure projects: hydroelectric dams and basic transportation needs such as bridges and highways.
\end{itemize}

\hypertarget{international-monetary-fund}{%
\subsection{International Monetary Fund}\label{international-monetary-fund}}

The International Monetary Fund (IMF) provides stability in exchange rates.

\begin{itemize}
\item
  Fixed exchange rates (before 1976)
\item
  Floating exchange rates (1976)

  \begin{itemize}
  \tightlist
  \item
    After the United States ended the convertibility of dollars to gold, monetary cooperation became the responsibility of the \textbf{Group of 7 (G7)}.
  \end{itemize}
\item
  Short-term loans to countries facing temporary crisis
\item
  Encouraging structural adjustments and providing policy advice on macroeconomic issues and economic restructuring
\end{itemize}

\hypertarget{general-agreement-on-tariffs-and-trade}{%
\subsection{General Agreement on Tariffs and Trade}\label{general-agreement-on-tariffs-and-trade}}

Enshrined liberal principles:

\begin{itemize}
\item
  Support of trade liberalization
\item
  Nondiscrimination in trade; most-favored-nation treatment
\item
  Preferential access in developed markets for products from the Global South
\item
  Support for the national treatment of foreign enterprises
\end{itemize}

\hypertarget{world-trade-organization}{%
\subsection{World Trade Organization}\label{world-trade-organization}}

In 1995, GATT became a formal institution, renaming itself the World Trade Organization (WTO).

\begin{itemize}
\item
  The World Trade Organization (WTO) incorporated the general areas of GATT's jurisdiction and expanded jurisdiction in services and intellectual property.
\item
  Regular ministerial meetings gave the WTO new political prominence.
\item
  Represents states that conduct over 90 percent of the world's trade.
\item
  Serves as a forum for trade negotiations.
\item
  Provides a venue for trade review, dispute settlement, and enforcement.
\end{itemize}

\hypertarget{international-monetary-policy}{%
\section{International Monetary Policy}\label{international-monetary-policy}}

During the 1920s and after World War II, the value of the U.S. dollar was linked to gold. In 1971, the dollar was taken off the gold standard. The prices of each currency adjust continually in response to market supply and demand. Currency trades average more than \$3 trillion a day. No global institution addressing monetary policy.

\begin{itemize}
\item
  Central role of the United States and the U.S. dollar as the reserve currency
\item
  The U.S. dollar serves as the world reserve currency, giving the United States enormous power.
\end{itemize}

\hypertarget{international-trade-negotiations}{%
\section{International Trade Negotiations}\label{international-trade-negotiations}}

1986: Uruguay Round of trade talks covered new areas, such as services, intellectual property rights, and agriculture.

The World Trade Organization (WTO) created the .p1-green{[}Trade Policy Review Mechanism{]} and .p1-red{[}Dispute Settlement Body{]}. Nonetheless, the WTO has proved to be a weak institution.

\begin{itemize}
\item
  Between 1947 and 1994, the parties in the GATT were successful in cutting tariffs, giving better treatment to developing countries, and addressing new problems (subsidies and countervailing duties).
\item
  Overall, tariffs were reduced in the major trading countries from an average of 40 percent to 5 percent on imported goods.
\end{itemize}

2001: the Doha Round began

\begin{itemize}
\item
  Main sticking point remained agriculture market liberalization in the United States and European Union
\item
  Failed over perceptions of trade fairness.
\end{itemize}

The WTO has proved to be a weak institution for facilitating that process, given the procedure adopted by negotiators: ``nothing is agreed until everything is agreed.''

\hypertarget{globalization}{%
\section{Globalization}\label{globalization}}

\hypertarget{definition}{%
\subsection{Definition}\label{definition}}

\begin{itemize}
\item
  As a \textbf{process}, the flows of goods, services, people, capital, and technology within a single world economy and the
  transformations of national economies that these flows produce.
\item
  As an \textbf{outcome}, a world economy in which government policies pose few barriers to, and technology enables, cross-border economic transactions.
\end{itemize}

\hypertarget{measurements}{%
\subsection{Measurements}\label{measurements}}

Trade openness: Sum of the total imports and exports as a share of GDP

\begin{itemize}
\tightlist
\item
  Trade openness = {[}imports + exports{]}/GDP
\end{itemize}

Capital flows: Total capital inflows and outflows as a share of GDP

\begin{itemize}
\tightlist
\item
  Consist of direct investment, portfolio, and bank finance
\end{itemize}

We can utilize other various indicators to see globalization

\begin{itemize}
\item
  Private capital flow (net or percentage of GDP)
\item
  Manipulated indicators of globalization (social/economic/political)
\item
  Foreign direct investment (inflow or outflow)
\end{itemize}

\hypertarget{ricardos-comparative-advantage}{%
\subsection{Ricardo's Comparative Advantage}\label{ricardos-comparative-advantage}}

Theory of comparative advantage

\begin{itemize}
\item
  Specialization and free trade are the best policies for countries.
\item
  They increase world production and the consumption possibility for each country by increasing the efficiency.
\item
  Ricardo emphasized opportunity costs while previous scholars studied absolute costs to calculate utilities.
\end{itemize}

\begin{table}

\caption{\label{tab:unnamed-chunk-5}Number of Hours Required to Produce 1 Unit of Goods}
\centering
\begin{tabu} to \linewidth {>{\raggedright\arraybackslash}p{5em}>{\raggedleft\arraybackslash}p{4em}>{\raggedleft\arraybackslash}p{4em}>{\raggedright}X>{\raggedright}X}{>{\raggedright\arraybackslash}p{5em}|>{\raggedleft\arraybackslash}p{4em}|>{\raggedleft\arraybackslash}p{4em}|l|l}
\hline
Country & Cloth & Wine & Opportunity cost of cloth & Opportunity cost of wine\\
\hline
England & 100 & 120 & 0.833 wine (100/120) & 1.200 cloth (120/100)\\
\hline
Portugal & 90 & 80 & 1.125 wine (90/80) & 0.888 cloth (80/90)\\
\hline
\end{tabu}
\end{table}

How much cloth is forgone in England when one unit of wine is produced?

\begin{itemize}
\item
  1.20 units of cloth
\item
  120 hours to produce one unit of wine/100hours to produce one unit of cloth.
\end{itemize}

Opportunity cost

\begin{itemize}
\item
  The internal trade-off in production between goods.
\item
  Choose goods for trade by the relative cost, not by the absolute cost.
\item
  In sum, trade makes countries better off

  \begin{itemize}
  \tightlist
  \item
    England gets 1 unit of wine from Portugal rather than having to pay 1.2 units of cloth to produce 1 unit of wine.
  \end{itemize}
\end{itemize}

\hypertarget{critique-of-ricardos}{%
\subsubsection{Critique of Ricardo's}\label{critique-of-ricardos}}

Specialization and free trade may not always result in more rapid progress over time, Why? Some goods can have varying values. In other words, some goods are more likely to benefit from the application of science and technology.

\begin{itemize}
\item
  Reduce their production costs over time.
\item
  Contribute to economic expansion.
\end{itemize}

For a country's future, what is crucial is not specialization \emph{per se}, but \textbf{the choice of what to specialize in}.

\hypertarget{stolper-samuelson-theorem}{%
\subsection{Stolper-Samuelson Theorem}\label{stolper-samuelson-theorem}}

An equilibrium model of international trade, building on Ricardo's comparative advantage. Relative endowments of the production factors (land, labor, and capital) determine a country's comparative advantage. Countries have comparative advantages in those goods required abundant factors.

\begin{itemize}
\item
  B/c goods with abundant factor are cheaper to produce \(\rightarrow\) Good for export
\item
  Goods with scarce factor \(\rightarrow\) Good for import
\end{itemize}

\hypertarget{differences-between-ricardo-and-stolper-samuelson}{%
\subsection{Differences between Ricardo and Stolper-Samuelson}\label{differences-between-ricardo-and-stolper-samuelson}}

Ricardo

\begin{itemize}
\item
  Comparative advantage by differences in labor productivity (technologies)
\item
  A single factor of production (labor)
\end{itemize}

Stolper-Samuelson

\begin{itemize}
\item
  Identical production technology everywhere
\item
  Relative ratio of capital, labor, and land
\end{itemize}

\begin{tabu} to \linewidth {>{\raggedright\arraybackslash}p{8em}>{\raggedright}X>{\raggedright}X}{>{\raggedright\arraybackslash}p{8em}|l|l}
\hline
  & Technology & Factor\\
\hline
Ricardo & Different & Only labor\\
\hline
Stolper-Samuelson & Identical & Capital, labor, and land\\
\hline
\end{tabu}

\hypertarget{main-prediction-of-stolper-samuelson-theorem}{%
\subsubsection{Main Prediction of Stolper-Samuelson Theorem}\label{main-prediction-of-stolper-samuelson-theorem}}

A country is better off exporting goods with abundant factors.

\begin{tabu} to \linewidth {>{\raggedright\arraybackslash}p{7em}>{\raggedright}X>{\raggedright}X}{>{\raggedright\arraybackslash}p{7em}|l|l}
\hline
  & Abundant & Factor\\
\hline
Developed & Skilled labor and capital & Unskilled labor\\
\hline
Developing & Unskilled labor & Skilled labor and capital\\
\hline
\end{tabu}

Trade benefits capital and harms labor in developed countries, \emph{vice versa}.

\begin{itemize}
\item
  In developed countries, inequality \(\uparrow\)
\item
  In developing countries, inequality \(\downarrow\)
\end{itemize}

\hypertarget{critique-of-the-theorem}{%
\subsubsection{Critique of the Theorem}\label{critique-of-the-theorem}}

Leontief Paradox

\begin{itemize}
\item
  SS theory predicts that each country exports the commodity which uses its abundant factor.
\item
  The first test by Wassily W. Leontief in 1954
\item
  US, the most capital abundant, exported labor-intensive goods and imported capital-intensive goods.
\end{itemize}

Lower inequality in developing countries?

\begin{itemize}
\tightlist
\item
  In Latin Ameerica, inequality is rising as trade increases.
\end{itemize}

\hypertarget{capital-mobility-theory}{%
\subsection{Capital Mobility Theory}\label{capital-mobility-theory}}

Impacts of international capital mobility

\begin{itemize}
\item
  Increase the power of capital over government that tries to expand social protection

  \begin{itemize}
  \item
    Mobile capital pursues profitable return on investment vs.~Governments compete to attract investment
  \item
    Mobile capital can threaten government by `exit option.'
  \end{itemize}
\end{itemize}

Impacts of international capital mobility

\begin{itemize}
\item
  Lower tax rates on mobile assets

  \begin{itemize}
  \tightlist
  \item
    Corporate profits and high income-earners.
  \end{itemize}
\item
  Capital exit's option weakens unions and social corporatism.
\end{itemize}

\hypertarget{compensation-theory}{%
\subsection{Compensation Theory}\label{compensation-theory}}

Historical association between economic openness and bigger public sector.

\begin{itemize}
\item
  He accepts that openness promotes efficiency.
\item
  Small countries gain the benefits of economies of scale.
\item
  However, economic restructuring has a negative impact on unskilled workers.
\end{itemize}

The welfare state works as a mechanism of domestic compensation.

\begin{itemize}
\item
  Welfare benefits: side-payments to the losers in market.
\item
  Social welfare may be an important political lubricant, facilitating social acceptance of economic change.
\end{itemize}

\hypertarget{empirical-findings}{%
\subsection{Empirical Findings}\label{empirical-findings}}

\begin{enumerate}
\def\labelenumi{\arabic{enumi}.}
\item
  Surprisingly, few empirical research on the impact of globalization on inequality.

  \begin{itemize}
  \tightlist
  \item
    Their findings are inconclusive.
  \end{itemize}
\item
  Studies of globaization and welfare spending in \emph{developed countries} are voluminious.

  \begin{itemize}
  \item
    Their empirical findings are still inconclusive.
  \item
    Due to, in part, different measurements \(\rightarrow\) Myriad ways to measure to the welfare state and globalization.
  \end{itemize}
\item
  Study of welfare state and globalization among \emph{developing countries} is of more recent vintage.

  \begin{itemize}
  \item
    But empirical findings may converge into a single point.
  \item
    Negative effect of globalization on welfare spending.
  \end{itemize}
\item
  Study of welfare state and globalization among \emph{developing countries} is of more recent vintage.

  \begin{itemize}
  \item
    Different patterns of integration into global markets
  \item
    Difference of the bargaining power of labor.
  \item
    Left parties differ.
  \end{itemize}
\end{enumerate}

\hypertarget{article-milner-1999-the-political-economy-of-international-trade}{%
\subsection{Article: Milner (1999) The Political Economy of International Trade}\label{article-milner-1999-the-political-economy-of-international-trade}}

Explores the existing theories about trade and trade policy, focusing on preferences(interests), institutions, and systems.

\begin{enumerate}
\def\labelenumi{\arabic{enumi}.}
\item
  Changes in trade policy preferences among domestic actors (groups, leaders).

  \begin{itemize}
  \item
    Economists: Free trade is the best policy for most countries most of the time.
  \item
    Political scientists: Pressure group model, the recourse to protection by governments as a function of the demands made by domestic group.
  \end{itemize}
\item
  Changes in political institutions to account for such policy change.

  \begin{itemize}
  \item
    Some institutions tend to give special interest groups greater access to policy makers, rendering their demands harder to resist.
  \item
    Other institutions insulate policy makers from these demands, allowing them more leeway in setting policy.
  \item
    Other aspects of political regime: electoral system/state capacity/Dem-Auto.
  \end{itemize}
\item
  Changes in the international political system.

  \begin{itemize}
  \tightlist
  \item
    IO between countries: (1) reduce transaction costs, (2) prevent countries from betrayal, (3) enjoy absolute gains.
  \end{itemize}
\item
  Reciprocal impact of trade on the domestic and international political system.

  \begin{itemize}
  \item
    Openness raises the potential number of supporters of free trade as exporters and multinational firms.
  \item
    According to the compensation theory, Openness \(\rightarrow\) Economic growth.
  \item
    Constraining influence of political conditions; PR system, welfare state regime
  \end{itemize}
\end{enumerate}

\hypertarget{strategies-to-achieve-economic-development}{%
\subsection{Strategies to Achieve Economic Development}\label{strategies-to-achieve-economic-development}}

Following World War II and the quick end to colonial rule in the 1950s and 1960s, international development focused on large infrastructure projects. In the 1970s, international development began funding projects on health, education, and housing to improve lives of the poor.

The 1980s saw a shift toward reliance on the private sector.

\begin{itemize}
\tightlist
\item
  \textbf{Washington Consensus}: privatization, liberalization of trade and foreign direct investment, broad tax reform, and deregulation are needed for development to occur.
\end{itemize}

The 1990s witnessed an emphasis on sustainable development.

\begin{itemize}
\tightlist
\item
  \textbf{World Bank}: Began promoting sustainable development with an emphasis that incorporates concern for renewable resources and the environment
\end{itemize}

The 1990s did see growth in emerging markets, and average per-capita incomes in the developing world have grown at a faster rate than in the developed economies.

\begin{itemize}
\item
  Millennium Development Goals (MDGs) were designed to reduce poverty by 2015 with eight goals; it had limited success.
\item
  In 2015, 17 Sustainable Development Goals (SDGs) were announced for 2030. Yet, the COVID-19 pandemic disrupted the implementation of these SDGs.
\end{itemize}

The \textbf{Beijing Consensus} emerged, with an emphasis on China's rapid, state-driven growth as an alternative model to development.

\begin{itemize}
\tightlist
\item
  Experimenting with policies that are compatible with a state's political structure and cultural experience.

  \begin{itemize}
  \tightlist
  \item
    This may include using capitalist tools as well as employing state-owned enterprises to invest capital in the country's own markets and abroad.
  \end{itemize}
\item
  This approach was viewed quite favorably because China continued to have high growth rates and had apparent success in weathering the global financial crisis and the 2020 pandemic-induced economic lockdown.
\end{itemize}

\hypertarget{economic-challenges}{%
\subsection{Economic Challenges}\label{economic-challenges}}

International crises have been a recurrent feature of the global economic system.

\begin{itemize}
\item
  Asian financial crisis (1997--99)
\item
  Global financial crisis (2008--2009)
\item
  The COVID-19 pandemic crisis
\end{itemize}

\hypertarget{covid-19-and-resulting-economic-crisis}{%
\subsubsection{COVID-19 and Resulting Economic Crisis}\label{covid-19-and-resulting-economic-crisis}}

The COVID-19 pandemic caused an economic crisis like no other:

\begin{itemize}
\item
  Countries locked down, leading to skyrocketing unemployment and collapsing trade flows.
\item
  Countries' GDPs dropped precipitously.
\item
  By the end of 2020, 38 governments were at risk of default.
\item
  Workers in the developing world were severely affected due to lack of safety nets, unemployment insurance, and savings.
\item
  Many developing countries have seen economic regression back to levels last seen in the 1990s.
\item
  Developing countries could not respond with the same monetary and fiscal policies as their developed counterparts in order to stimulate their economies.
\end{itemize}

\hypertarget{responses-to-economic-crises}{%
\subsubsection{Responses to Economic Crises}\label{responses-to-economic-crises}}

The global financial crisis and the Eurozone crisis prompted reforms.

\begin{itemize}
\item
  Economic liberals believe these to be sufficient.
\item
  But the \emph{moral hazard} is not alleviated: states rescued from the consequences of their reckless behavior may have little incentive to change that behavior in the future.
\end{itemize}

Minimizing the health and economic impacts of the COVID-19 pandemic and future pandemics is more challenging.

\begin{itemize}
\item
  Investment in disease surveillance systems and pandemic preparedness is needed.
\item
  Problems: distributing economic burdens across countries and economic nationalism
\end{itemize}

\hypertarget{development-and-human-security}{%
\chapter{Development and Human Security}\label{development-and-human-security}}

\hypertarget{development}{%
\section{Development}\label{development}}

\hypertarget{what-doesnt-explain-major-differences-in-poverty-and-prosperity}{%
\subsection{What doesn't explain major differences in poverty and prosperity?}\label{what-doesnt-explain-major-differences-in-poverty-and-prosperity}}

\begin{enumerate}
\def\labelenumi{\arabic{enumi}.}
\item
  Geography/climate
\item
  Culture
\item
  Colonialism
\item
  Ignorance Hypothesis
\end{enumerate}

\hypertarget{inclusive-vs-extractive-economic-institutions-acemoglu-and-robinson-why-nations-fail}{%
\subsection{\texorpdfstring{Inclusive vs extractive economic institutions (Acemoglu and Robinson, \emph{Why Nations Fail})}{Inclusive vs extractive economic institutions (Acemoglu and Robinson, Why Nations Fail)}}\label{inclusive-vs-extractive-economic-institutions-acemoglu-and-robinson-why-nations-fail}}

\begin{itemize}
\item
  Inclusive economic inst. encourage people to participate in society and allow them to benefit from their work and choose the type of interests, education, and jobs they want to have (p.~74)
\item
  Private property, unbias system of law, Provision of public services available to everyone
\item
  Paves the way for technology adaptation and education for the masses
\end{itemize}

\hypertarget{inclusive-vs-extractive-political-institutions}{%
\subsubsection{Inclusive vs extractive political institutions}\label{inclusive-vs-extractive-political-institutions}}

\begin{itemize}
\item
  Extractive political institutions, concentrate power to a small elite who then create further institutions that restrict voting and political freedom
\item
  Interact with extractive economic institutions that take resources from the rest of society to enrich themselves
\item
  ``inclusive political institutions, vesting power broadly, would tend to uproot economist institutions that expropriate the resources of the many, erect entry barriers, and suppress the functioning of markets'' (p.~81)
\end{itemize}

\hypertarget{washington-consensus-market-centric-approach-vs.-beijing-consensus-government-led-approach}{%
\subsection{Washington Consensus (Market centric approach) vs.~Beijing Consensus (Government-led approach)}\label{washington-consensus-market-centric-approach-vs.-beijing-consensus-government-led-approach}}

\hypertarget{human-security}{%
\section{Human Security}\label{human-security}}

Aims to address ``widespread and cross-cutting challenges to the survival, likelihood and dignity of \ldots{} people.''

\begin{itemize}
\tightlist
\item
  According to the UN General Assembly resolution, human security also encompasses ``the right of people to live in freedom and dignity, free from poverty and despair.
\end{itemize}

\hypertarget{population-dynamics}{%
\subsection{Population Dynamics}\label{population-dynamics}}

\hypertarget{malthusian-dilemma}{%
\subsubsection{Malthusian dilemma}\label{malthusian-dilemma}}

Malthus argued that population increase outstrips the food supply, threatening human security. According to him, resource supply increases in linear pattern while population increases geometrically. However, his theory did not take into account demographic transition.

\begin{itemize}
\item
  Population increase is not uniformly distributed: poorer countries have higher growth rates.
\item
  Population growth and economic development mean more demand for scarce natural resources, in particular, arable land and freshwater.
\item
  Lowering population growth rates leads to ethical dilemmas.
\end{itemize}

\hypertarget{decline-of-population}{%
\subsubsection{Decline of Population}\label{decline-of-population}}

\begin{itemize}
\item
  Populations of many Western states are declining and aging.
\item
  Aging populations represent a financial burden for their working counterparts.
\item
  This slows economic growth and weakens state power.
\end{itemize}

\hypertarget{migration}{%
\subsection{Migration}\label{migration}}

\hypertarget{refugees}{%
\subsubsection{Refugees}\label{refugees}}

\begin{itemize}
\item
  Displaced people who cannot safely return home due to a well-founded fear of being persecuted.
\item
  UN High Commissioner for Refugees is responsible for protecting refugees until they are granted asylum or return home.
\item
  Refugees are entitled to non-refoulement.

  \begin{itemize}
  \item
    Refugees cannot be forced to return to their country of origin.
  \item
    Broader definition includes people fleeing civil war and climate change.
  \end{itemize}
\end{itemize}

\hypertarget{internally-displaced-people-idps}{%
\subsubsection{Internally displaced people (IDPs)}\label{internally-displaced-people-idps}}

\begin{itemize}
\item
  People who have been uprooted from their homes but remain in their home country.

  \begin{itemize}
  \item
    Technically still under their state's protection
  \item
    The state is failing at its basic responsibilities.
  \end{itemize}
\end{itemize}

\hypertarget{economic-migrants}{%
\subsubsection{Economic migrants}\label{economic-migrants}}

\begin{itemize}
\item
  People fleeing poverty, unemployment, poor economic prospects, etc.
\item
  Don't have legal right to asylum.
\item
  Are not entitled to international legal protection or to non-refoulement (protection gap)
\end{itemize}

\hypertarget{the-primary-difference-between-a-refugee-and-an-asylee}{%
\subsubsection{The Primary Difference between a Refugee and an Asylee}\label{the-primary-difference-between-a-refugee-and-an-asylee}}

Not every asylum seeker will ultimately be recognised as a refugee, but every refugee is initially an asylum seeker.

\hypertarget{crisis-in-numbers}{%
\subsubsection{Crisis in Numbers}\label{crisis-in-numbers}}

The international community took a traditional approach in providing ``temporary'' camps for the displaced, an approach widely used to address the Palestinian refugee crisis in the 1950s.

\hypertarget{global-health}{%
\subsection{Global Health}\label{global-health}}

Public health and communicable disease are issues that have threatened human security across all types of geographic borders throughout human history. Twenty-first-century mobility results in the easier spread of many communicable diseases. The WHO obligates states to report new outbreaks within 24 hours and also provides technical assistance to developing countries.

\hypertarget{sars-ebola-and-covid-19}{%
\subsubsection{SARS, Ebola, and COVID-19}\label{sars-ebola-and-covid-19}}

The SARS (severe acute respiratory syndrome) outbreak of 2002--2003 illustrated the urgency of quick and honest reporting.

\begin{itemize}
\tightlist
\item
  China's suppression of information proved detrimental.
\end{itemize}

The 2014 Ebola outbreak in West Africa illustrated the failures of both states and the WHO.

\begin{itemize}
\item
  Neither the WHO nor the states were in charge.
\item
  NGOs (Doctors Without Borders) organized assistance.
\item
  Over 11,300 people died; economies affected.
\end{itemize}

COVID-19 was first detected on Dec.~19, 2019, in Wuhan, China.

\begin{itemize}
\item
  China confirmed human-to-human transmission on Jan.~22, 2020.
\item
  China suppressed information and failed to warn the world.
\item
  The WHO failed to provide timely warnings, waiting until March 11, 2020, to declare a global pandemic.
\item
  Individual states were unprepared and, many, failed to report numbers, making tracing difficult.
\item
  The developing world received little of the vaccine supply.
\item
  The COVID-19 pandemic continues to have grave economic repercussions.
\item
  The world may become more protectionist and less open (Stephen Walt).
\item
  Economic inequality and regional differences are likely to increase.
\end{itemize}

\hypertarget{the-impact-of-human-security-issues-on-the-study-of-international-relations}{%
\subsection{The Impact of Human Security Issues on the Study of International Relations}\label{the-impact-of-human-security-issues-on-the-study-of-international-relations}}

\begin{itemize}
\item
  Human security issues, which are transnational in nature, force us to rethink major areas of international relations.
\item
  State sovereignty: transnational problems and the rise of nonstate actors challenge state sovereignty.
\item
  Future of globalization: Will human security issues increase multilateralism and globalization?
\item
  Theoretical frameworks: long-standing assumptions often require rethinking.
\end{itemize}

\hypertarget{energy-politics-and-climate-negotiations}{%
\chapter{Energy Politics and Climate Negotiations}\label{energy-politics-and-climate-negotiations}}

\hypertarget{the-environment-protecting-the-global-commons}{%
\section{The Environment: Protecting the Global Commons}\label{the-environment-protecting-the-global-commons}}

\hypertarget{two-conceptual-perspectives}{%
\subsection{Two conceptual perspectives}\label{two-conceptual-perspectives}}

\begin{itemize}
\item
  Collective goods: need to achieve shared benefits by overcoming conflicting interests; Hardin's concept of the commons

  \begin{itemize}
  \item
    The planet itself is a commons; its resources are finite.
  \item
    Sustainability: policies that promote change that neither damages the environment nor depletes finite resources
  \item
    Advancing our survival without doing lasting damage
  \item
    Costs of environmental harm are diffused across both space and time.
  \item
    Free riding

    \begin{itemize}
    \item
      The costs of harm to the environment are diffused across both space and time, and the benefits of pollution and unsustainable resource consumption are concentrated and immediate.
    \item
      Each individual state, corporation, or person has a strong incentive to enjoy a ``free ride'' and hope others will bear the costs of restraint.
    \item
      Free riding is also what we see in collective action problems and tragedies of the commons.
    \end{itemize}
  \end{itemize}
\end{itemize}

\hypertarget{the-environment-as-an-issue-in-international-relations}{%
\subsection{The Environment as an Issue in International Relations}\label{the-environment-as-an-issue-in-international-relations}}

Degradation of the environment emerged as an issue in international relations in the twentieth century. Human activity produces negative externalities.

\begin{itemize}
\item
  Definition: A side effect or consequence of an industrial or commercial activity that affects other parties without this being reflected in the cost of the goods or services involved.
\item
  Example: Pesticides that help us reduce the spread of diseases devastate animal reproductive cycles.
\end{itemize}

\hypertarget{institutionalizing-environmental-protection}{%
\subsection{Institutionalizing Environmental Protection}\label{institutionalizing-environmental-protection}}

A series of soft laws developed to protect the environment, which are not legally binding but may become so in the future.

\begin{itemize}
\item
  \emph{No significant harm principle}: A state cannot initiate policies that cause significant environmental damages to another state.
\item
  \emph{Good neighbor principle}: States should take care to avoid acts or omissions that could reasonably be foreseen to cause harm to neighboring states.
\item
  \emph{Polluter pays principle}: Those causing the pollution should be responsible for cleaning it up, or curtailing it.
\item
  \emph{Precautionary principle}: Action should be taken based on scientific warning before irreversible harm occurs.
\item
  \emph{Preventive action principle}: States should take action in their own jurisdictions to avoid harm to the environment.
\end{itemize}

In addition to states, IGOs and NGOs play prominent roles in protecting the global commons.

\begin{itemize}
\item
  Publicize their dissatisfaction
\item
  Get environmental issues onto international agendas and influence state behavior
\item
  Function as part of epistemic communities

  \begin{itemize}
  \item
    A transnational community of experts and technical specialists from international organizations, NGOs, and state and substate agencies that share a set of beliefs.
  \item
    These communities share expertise, notions of validity, and a set of practices organized around solving a particular problem.
  \end{itemize}
\end{itemize}

Trade agreements are another institutional way to address environmental problems.

\begin{itemize}
\item
  The North American Free Trade Agreement (NAFTA) instituted in 1995 was known at first as more environmentally friendly than other agreements.
\item
  It required each party to maintain its own level of environmental protection and ban imports produced in violation of those standards.
\end{itemize}

Environmental problems can be managed if states agree to cooperate.

\begin{itemize}
\item
  Example: ozone depletion due to use of chlorofluorocarbons (CFCs)

  \begin{itemize}
  \item
    States agreed to phase out the use of CFCs.
  \item
    Developed states agreed to finance costs of compliance.
  \item
    Multinational corporations (MNCs) eventually supported the prohibition of CFCs.
  \end{itemize}
\end{itemize}

\hypertarget{the-problem-of-climate-change}{%
\section{The Problem of Climate Change}\label{the-problem-of-climate-change}}

The issue of climate change is complicated as there are no inexpensive substitutes for agricultural, communications, and industrial processes.

Scientific facts are indisputable: the earth is warming.

\begin{itemize}
\item
  With a projected increase of between 1.9 and 3 degrees Celsius estimated by the end of the twenty-first century.
\item
  That acceleration in atmospheric and ocean warming has resulted in glacial and ice sheet melting and a rise in sea levels, as well as rising temperatures on land and the exponential increase in the number of extreme weather events like hurricanes, floods, and fires.
\end{itemize}

Climate change has a high human cost.

\begin{itemize}
\item
  Low elevations losing land mass
\item
  Extreme weather events
\item
  Higher rates of waterborne diseases
\end{itemize}

Costs of solutions are immediate, but benefits emerge after decades.

\hypertarget{approaches-to-climate-change-mitigation}{%
\subsection{Approaches to Climate Change Mitigation}\label{approaches-to-climate-change-mitigation}}

The international community has made attempts to respond to climate change through \textbf{mitigation}: reducing greenhouse gas emissions and enhancing carbon sinks.

\begin{itemize}
\item
  The Kyoto Protocol (1997)

  \begin{itemize}
  \item
    Developed states agree to reduce emissions
  \item
    Less developed states are not obligated.
  \item
    Provides flexible mechanisms to meet goals
  \item
    Trading of emissions shares

    \begin{itemize}
    \item
      Credits earned from carbon sinks
    \item
      Credit for aiding other states in meeting standards
    \end{itemize}
  \end{itemize}
\item
  The Paris Agreement (2015)
\item
  Getting top polluters (the United States, China, and India) to limit emissions
\item
  \textbf{Geoengineering}: large-scale manipulations of the physical, chemical, or biological systems to reduce levels of atmospheric gas
\end{itemize}

\hypertarget{climate-change-adaptation}{%
\subsection{Climate Change Adaptation}\label{climate-change-adaptation}}

\textbf{Adaptation} to climate change means shifting resources into preparing for and remediating the effects of climate change.

\begin{itemize}
\item
  Using scarce resources more efficiently
\item
  Using drought resistant crops
\item
  Achieving energy efficiency
\item
  Adopting stronger building codes to prepare for extreme weather events
\end{itemize}

Adaptation is very costly; the costs are immediate whereas the benefits are not.

\hypertarget{article-green-and-hale-2017}{%
\subsection{Article: Green and Hale (2017)}\label{article-green-and-hale-2017}}

Majority of IR scholars find climate change among the top three most important policy issues, yet fewer than 4\% identify the environment as their primary area of research. It is a short paper with diagnosis and suggestions.

Green and Hale (2017) argue that greater attention to environmental issues in IR can bring significant benefits to the discipline. Also, they discuss solutions to correct this imbalance.

\begin{itemize}
\item
  Build \emph{environmental politics} as a discipline.
\item
  Bringing GEP (global environmental politics) into the IR mainstream
\end{itemize}

\hypertarget{natural-resource-issues}{%
\subsection{Natural Resource Issues}\label{natural-resource-issues}}

Natural resource issues are about increasing demand and declining resources.

\hypertarget{freshwater-resources}{%
\subsubsection{Freshwater Resources}\label{freshwater-resources}}

\begin{itemize}
\item
  As freshwater becomes more scarce, freshwater has become a major international issue.
\item
  Can lead to conflicts (e.g., Middle East, Ethiopia's Grand Resistance Dam)
\end{itemize}

\hypertarget{land-resources-forest}{%
\subsubsection{Land Resources: Forest}\label{land-resources-forest}}

\begin{itemize}
\item
  The rate of deforestation has been very high: In 2019, 12 million hectares of tropical tree cover were lost; even more in 2020.
\item
  Deforestation has significant consequences.
\end{itemize}

\hypertarget{protection-of-species}{%
\subsubsection{Protection of Species}\label{protection-of-species}}

\begin{itemize}
\item
  Almost one million animal and plant species are being threatened with extinction within decades. Various conventions and treaties try to prevent that.
\item
  Convention on Biological Diversity (1992) aims to conserve biodiversity and advocates equitable and sustainable use.
\end{itemize}

\hypertarget{polution-of-the-commons}{%
\subsection{Polution of the Commons}\label{polution-of-the-commons}}

\hypertarget{air}{%
\subsubsection{Air}\label{air}}

Every year, thousands of deaths can be attributed to air pollution.

In most instances, extreme air pollution is treated domestically.

\begin{itemize}
\tightlist
\item
  Yet, pollution does not respect national borders.
\end{itemize}

\hypertarget{oceans}{%
\subsubsection{Oceans}\label{oceans}}

Oceans are warming and absorb excess carbon dioxide making seawater more acidic and less oxygenated.

May move through the food chain and affect humans

Agricultural activities affect coastal waters. Fertilizers can change water's acidity and oxygen levels.

\hypertarget{environmental-issues-and-conflict}{%
\subsection{Environmental Issues and Conflict}\label{environmental-issues-and-conflict}}

The degradation of renewable resources may lead to violence as people start competing over them.

\begin{itemize}
\tightlist
\item
  Especially true about oil, which is vital for industry and economy
\end{itemize}

Climate change may also lead to insecurity and violence.

Climate change may also exacerbate migration, often to contested lands.

\hypertarget{final-exam-pool}{%
\chapter*{Final Exam Pool}\label{final-exam-pool}}
\addcontentsline{toc}{chapter}{Final Exam Pool}

It is a living documents by April 22th.

\hypertarget{part-i.-multiple-choice}{%
\section*{Part I. Multiple-Choice}\label{part-i.-multiple-choice}}
\addcontentsline{toc}{section}{Part I. Multiple-Choice}

I will pick 15 MC questions from here. 3 MC questions will come from BDM et al.~(2005), Milner (1999), Green and Hale (2017). Please read their \textbf{abstracts} carefully.

\begin{center}\rule{0.5\linewidth}{0.5pt}\end{center}

\begin{enumerate}
\def\labelenumi{\arabic{enumi}.}
\item
  According to realists, cooperation is difficult because?

  A. states' lack of shared identities

  B. states' concern with relative gains

  C. transaction costs associated with negotiating treaties

  D. reciprocity
\item
  The prisoner's dilemma reflects the problem of

  A. distribution problems.

  B. norms of noncooperation.

  C. socialization.

  D. cheating.
\item
  Which of the following is true of the realist view of cooperation?

  A. Realists believe that reciprocity prevents cooperation.

  B. Realists believe that continuous interactions can help states overcome the relative gains problem.

  C. Realists believe that noncooperation is the rational choice for states.

  D. Realists believe that both states are better off if they both do not cooperate than if they both cooperate.
\item
  Which of the following is true of the neoliberal institutionalist approach to the study of cooperation?

  A. Unlike realists, neoliberal institutionalists do not believe the international system is anarchiC.

  B. Unlike realists, neoliberal institutionalists do not view states as unitary actors.

  C. Like realists, neoliberal institutionalists think states will not cooperate because they are rational actors.

  D. Like realists, neoliberal institutionalists view states as rational actors, but believe that can lead to cooperation rather than noncooperation.
\item
  According to liberal trade theory, the fact that the General Agreement on Tariffs and Trade (GATT) creates a forum for multilateral negotiations

  A. reduces transaction costs, making cooperation more likely.

  B. reduces transaction costs, making cooperation less likely.

  C. increases transaction costs, making cooperation more likely.

  D. increases transaction costs, making cooperation less likely.
\item
  A body of rules and norms regulating interactions among states is known as

  A. international interdependence.

  B. the international constitution.

  C. international order.

  D. international law.
\item
  Which of the following factors drive horizontal enforcement of international law?

  A. reciprocity and legitimacy

  B. power and legitimacy

  C. power and reciprocity

  D. international courts
\item
  Which economic policy would a mercantilist favor?

  A. promoting exports over imports

  B. free trade

  C. promoting equitable distribution of resources in the international system

  D. promoting an increase in imports
\item
  The three Bretton Woods institutions are

  A. the United Nations, the General Agreement on Tariffs and Trade, and the World Bank.

  B. the United Nations, the General Agreement on Tariffs and Trade, and the World Trade Organization.

  C. the General Agreement on Tariffs and Trade, the World Bank, and the International Monetary FunD.

  D. the World Trade Organization, the United Nations, and the World Bank.
\item
  Based on the recognition that states differ in their resource endowments of land, labor, and capital, a theory developed arguing that states should trade based on their \_\_\_\_\_\_\_\_\_\_, whereby each state produces and exports those products that it can produce most efficiently relative to other states.

  A. exchange rate

  B. absolute advantage

  C. comparative advantage

  D. national interest
\item
  The liberal approach to economic development embodied by the Washington Consensus prioritizes which of the following policies?

  A. export subsidies

  B. employing state-owned enterprises

  C. privatization

  D. protectionism
\item
  Which of the following is true of the League of Nations?

  A. It was founded based on the idea of collective security.

  B. It was formed after World War II, and successfully prevented war until it ceded power to the United Nations in the 1960s.

  C. It was the successor to the United Nations.

  D. The United States was the first to join the organization.
\item
  Which of the following is a key difference between the League of Nations and the UN?

  A. The UN can authorize the use of force against a member state while the League of Nations cannot.

  B. The UN embodies the logic of collective security while the League of Nations did not.

  C. The United States oversaw the governance of the League of Nations while the UN is not governed by any one member state alone.

  D. UN actions focus on a broader view of security than the League of Nations did.
\item
  \_\_\_\_\_\_\_\_\_\_ is characterized by the UN seeking to contain conflicts between two states through third-party military forces, whereas \_\_\_\_\_\_\_\_\_\_ is characterized by both military and nonmilitary functions.

  A. Traditional peacekeeping; complex peacekeeping

  B. Non-refoulement; complex peacekeeping

  C. Traditional peacekeeping; non-refoulement

  D. Complex peacekeeping; traditional peacekeeping
\item
  Which policy has been the tool most used by the UN Security Council to encource peace?

  A. use of force

  B. non-refoulement

  C. economic sanctions

  D. preferential trade agreements
\item
  What are nongovernmental organizations (NGOs)?

  A. local and regional units of states

  B. subunits of the United Nations

  C. international institutions composed of states

  D. private, voluntary organizations whose members come together to achieve a common purpose
\item
  Which of the following is a source of NGOs' power in the international system?

  A. their rigid structure

  B. new communication technologies that allow them to build coalitions

  C. their voting power in the United Nations

  D. their consistent access to funding
\item
  Which of the following entities are the primary actors at the grassroots level in mobilizing individuals on political issues?

  A. IGOs

  B. NGOs

  C. state governments

  D. MNCs
\item
  Liberal political theorists place emphasis on \_\_\_\_\_\_\_\_\_\_, as laid out in documents such as the Bill of Rights.

  A. social and economic rights

  B. civil and political rights

  C. group rights for marginalized people

  D. collective rights such as the right to development
\item
  Which of the following is an example of civil and political rights?

  A. freedom from incarceration

  B. state sovereignty

  C. rejection of government policies that cause suffering

  D. freedom of speech
\item
  The International Bill of Rights consists of

  A. the Geneva Conventions and the United Nations Charter.

  B. only the Universal Declaration of Human Rights.

  C. only the International Covenant on Civil and Political Rights and the International Covenant on Economic, Social, and Cultural Rights.

  D. the Universal Declaration of Human Rights, the International Covenant on Civil and Political Rights, and the International Covenant on Economic, Social, and Cultural Rights.
\item
  In general, states with what government type are the most likely to abuse political and civil rights?

  A. liberal democracies

  B. autocracies

  C. presidential democracies

  D. parliamentary democracies
\item
  Early United Nations efforts following the 1949 Universal Declaration of Human Rights focused on getting states to grant women

  A. civil and political rights.

  B. socioeconomic rights.

  C. workers' rights.

  D. religious rights.
\item
  Which of the following is true about the status of women in the international community?

  A. Discrimination against women in political and public life is considered illegal by international treaties.

  B. NGOs have stopped pushing the issue of women's rights.

  C. Rape is only considered to be a wartime issue in international law.

  D. No international standards against trafficking in women exist.
\item
  Military action by states or the international community to alleviate massive human rights violations is known as

  A. genocidal relief.

  B. cultural relativist intervention.

  C. humanitarian intervention.

  D. intrastate war.
\item
  Which of the following statements is consistent with the idea of the ``responsibility to protect (R2P)''?

  A. States must intervene militarily to protect human rights when massive violations are occurring in another state.

  B. The UN Security Council must authorize military intervention when massive human rights violations are occurring in a state.

  C. States must intervene to protect human rights in the face of massive violations in another state, but they can never do so militarily.

  D. States must intervene when massive human rights violations are occurring in another state, but may only use military force when authorized by the UN Security Council.
\item
  \texttt{Milner\ (1999)}: One of the most salient changes in the world economy since 1980 has been the move toward \_\_\_\_\_\_\_\_\_\_ among countries across the globe. How do existing theories about trade policy explain this puzzle? Three sets of explanations are prominent. First, many focus on changes in trade policy preferences among \_\_\_\_\_\_\_\_\_\_, either societal groups or political leaders. Second, scholars examine changes in political institutions to account for such policy change. Third, they seek explanations in changes in the \_\_\_\_\_\_\_\_\_\_\_\_\_\_\_\_\_\_\_\_\_\_\_\_\_\_\_\_\_\_. Large-scale changes in political institutions, especially in the direction of democracy, may be necessary for the kind of massive trade liberalization that has occurred. But changes in preferences cannot be overlooked in explaining the rush to free trade. Moreover, the influence of international institutions has been important. Finally, the reciprocal impact of trade on domestic politics and the international political system is important. If the rush to free trade is sustained, will its impact be benign or malign?

  A. freer trade; international actors; domestic political system

  B. more protectionist trade; domestic actors; domestic political system

  C. freer trade; domestic actors; international political system

  D. freer trade; transnational actors; domestic political system
\item
  \texttt{Bueno\ de\ Mesquita\ et\ al.\ (2005)}: Research on human rights consistently points to the importance of democracy in reducing the severity and incidence of personal integrity abuses. The prescriptive implications of this finding for policy makers interested in state building have been somewhat limited, however, by a reliance on \_\_\_\_\_\_\_\_\_\_\_\_\_\_\_\_\_\_\_\_ of democracy. Consequently, a policy maker emerges from this literature confident that ``democracy matters'' but unclear about which set(s) of reforms is likely to yield a greater human rights payoff. Using data from the Polity IV Project, we examine what aspects of democracy are most consequential in improving a state's human rights record. Analysis of democracy's dimensions elicits three findings. First, political participation at the level of \_\_\_\_\_\_\_\_\_\_\_\_\_\_\_\_\_\_\_\_ appears more significant than other dimensions in reducing human rights abuses. Second, improvements in a state's level of democracy short of full democracy do not promote greater respect for integrity rights. Only those states with the highest levels of democracy, not simply those conventionally defined as democratic, are correlated with better human rights practices. Third, \_\_\_\_\_\_\_\_\_\_\_\_\_\_\_\_ appears to be the critical feature that makes full-fledged democracies respect human rights; limited accountability generally retards improvement in human rights.

  A. multidimensional measures; unconventional political arena; accountability

  B. unidimensional measures; multiparty competition; contestation

  C. unidimensional measures; unconventional political arena; contestation

  D. multidimensional measures; multiparty competition; accountability
\item
  \texttt{Green\ and\ Hale\ (2017)}: Despite the \_\_\_\_\_\_\_\_\_\_\_ urgency of many environmental problems, environmental politics remains at the margins of the discipline. Using data from the Teaching, Research, and International Policy (TRIP) project, this article identifies a puzzle: the majority of international relations (IR) scholars find \_\_\_\_\_\_\_\_\_\_\_ among the top three most important policy issues today, yet fewer than 4\% identify the environment as their primary area of research. Moreover, environmental research is rarely published in top IR journals, although there has been a recent surge in work focused on climate change. The authors argue that \_\_\_\_\_\_\_\_\_\_\_ attention to environmental issues---including those beyond climate change---in IR can bring significant benefits to the discipline, and they discuss three lines of research to correct this imbalance.

  A. increasing; climate change; greater

  B. increasing; economic growth; less

  C. decreasing; climate change; greater

  D. decreasing; economic growth; less
\end{enumerate}

\hypertarget{part-ii.-truefalse-question}{%
\section*{Part II. True/False Question}\label{part-ii.-truefalse-question}}
\addcontentsline{toc}{section}{Part II. True/False Question}

I will pick 15 T/F quizzes from here.

\begin{center}\rule{0.5\linewidth}{0.5pt}\end{center}

\begin{enumerate}
\def\labelenumi{\arabic{enumi}.}
\item
  Realists believe that cooperation is difficult because states care about absolute gains.

  A. True

  B. False
\item
  Realists only expect states to cooperate if they can both benefit in absolute terms.

  A. True

  B. False
\item
  Neoliberal institutionalists believe that the possibility of reciprocity in repeated interactions can foster cooperation among states.

  A. True

  B. False
\item
  Neoliberal institutionalists think international institutions are important in fostering cooperation, but other liberal theorists do not.

  A. True

  B. False
\item
  States that have ratified a treaty are legally bound by that treaty only after it enters into force.

  A. True

  B. False
\item
  There are no limitations on a state's actions in its territorial waters.

  A. True

  B. False
\item
  To supporters of economic liberalism, MNCs are a positive development.

  A. True

  B. False
\item
  China is now a member of the World Trade Organization.

  A. True

  B. False
\item
  While they differ in their arguments about free trade, economic liberals and economic nationalists have similar approaches to economic development.

  A. True

  B. False
\item
  The economic liberal approach to development includes privatization.

  A. True

  B. False
\item
  The manner and extent to which different IGOs carry out functions varies.

  A. True

  B. False
\item
  In the United Nations, a tension exists between addressing issues of human security and the need to respect state sovereignty.

  A. True

  B. False
\item
  Different NGOs often work at on differing purposes, resulting in countervailing pressures on policy issues.

  A. True

  B. False
\item
  Like liberals, realists believe NGOs play an important role in international relations.

  A. True

  B. False
\item
  Constructivists do not think that IGOs and NGOs can exert much influence over states.

  A. True

  B. False
\item
  Less developed states, even liberal democratic ones, often fail to meet obligations of social and economic rights.

  A. True

  B. False
\item
  The ``responsibility to protect (R2P)'' norm has been invoked only selectively.

  A. True

  B. False
\item
  Several states have withdrawn, or threatened to withdraw from, the ICC.

  A. True

  B. False
\item
  One key reason why environmental issues are difficult to resolve is because there is an incentive to ``free ride'' and hope others will bear the costs of adopting environmental protections.

  A. True

  B. False
\item
  Under international law, refugees can be forced to return to their country of origin.

  A. True

  B. False
\item
  Africa is where the most migrants are on the move and most migrants are hosted.

  A. True

  B. False
\item
  Some individuals and states benefit from the migration crisis.

  A. True

  B. False
\item
  States and IGOs have been unable to coordinate cooperative efforts to aid in the distribution of the COVID-19 vaccines across the world.

  A. True

  B. False
\item
  Liberals and realists agree on how to address health issues. They argue that only individual states are responsible for dealing with health issues when they threaten state security.

  A. True

  B. False
\item
  Inclusive political and economic institutions are those which do not provide private property.

  A. True

  B. False
\item
  Extractive political institutions allow for the major of the public to vote and no not concentrate power among the elite

  A. True

  B. False
\item
  Geography provides a comprehensive account for why nations have not been able to achieve economic prosperity

  A. True

  B. False
\end{enumerate}

\hypertarget{part-iii.-short-essay}{%
\section*{Part III. Short Essay}\label{part-iii.-short-essay}}
\addcontentsline{toc}{section}{Part III. Short Essay}

\begin{quote}
\emph{Instructions}: Write an essay that answers the question below. Your responses should be concise (in terms of content) and clear (in terms of writing). your response should not exceed more than a single sided page. Your answer is worth 10 points. (\(1 \times 10\) = 10 total points)
\end{quote}

\begin{center}\rule{0.5\linewidth}{0.5pt}\end{center}

Most economists have argued for many years that reducing barriers to international trade, often referred to as ``globalization,'' can in principle make everyone better off because of the theory of comparative advantage. These economists maintain that the world economy has grown over the last two decades due to globalization. Yet there has been a political backlash against globalization in many countries. What is the theory of comparative advantage and how does it work? Why has there been a backlash against globalization? According to the Stolper-Samuelson theory, which people in the US should be most opposed to globalization, and which people in India should be most opposed to globalization?

  \bibliography{book.bib,packages.bib}

\end{document}
